% Two-sided document
\documentclass[12pt, twoside]{ociamthesis}
\usepackage{helvet}
\usepackage{amssymb}
\usepackage[utf8]{inputenc}
\usepackage[T1]{polski}
\usepackage{url}
\usepackage[hidelinks]{hyperref}
\usepackage[table,xcdraw]{xcolor}
\usepackage[leftbars]{changebar}
\hypersetup{
	pdftitle={Detekcja choroby Alzheimera i stopnia demencji z użyciem narzędzi uczenia maszynowego w środowisku .NET},
	pdfsubject={Praca magisterska},
	pdfauthor={Mateusz Piotr Moruś},
	pdfkeywords={Choroba Alzheimera, demencja, uczenie maszynowe, dotnet, CSharp}
}
\usepackage{graphicx}
\usepackage{listings}
\usepackage{xcolor}

\definecolor{codegreen}{rgb}{0,0.6,0}
\definecolor{codegray}{rgb}{0.5,0.5,0.5}
\definecolor{backcolour}{rgb}{0.95,0.95,0.92}

\lstdefinestyle{mystyle}{
    backgroundcolor=\color{backcolour},
    commentstyle=\color{codegreen},
    keywordstyle=\color{magenta},
    numberstyle=\tiny\color{codegray},
    stringstyle=\color{codegreen},
    basicstyle=\ttfamily\footnotesize,
    breakatwhitespace=false,
    breaklines=true,
    captionpos=b,
    keepspaces=true,
    numbers=left,
    numbersep=5pt,
    showspaces=false,
    showstringspaces=false,
    showtabs=false,
    tabsize=4
}

\lstset{style=mystyle}


\setlength{\changebarsep}{-5em}

\graphicspath{ {./Images/} }
\AtBeginDocument{\let\textlabel\label}

% Make the numbering of figures continuous
\counterwithout{figure}{chapter}
% Make the numbering of tables continuous
\counterwithout{table}{chapter}

\title{Detekcja choroby Alzheimera i stopnia demencji z użyciem narzędzi uczenia maszynowego w środowisku .NET}
\engtitle{Detection of Alzheimer's disease and dementia severity using machine learning tools in .NET environment}
\author{Mateusz Piotr Moruś}
\albumnumber{292540}
\university{UNIWERSYTET MARII CURIE-SKŁODOWSKIEJ W~LUBLINIE}
\college{Wydział Matematyki, Fizyki i~Informatyki}
\field{Informatyka}
\speciality{-}
\submittedtext{Praca magisterska}
\department{Katedrze Neuroinformatyki i~Inżynierii Biomedycznej}
\promoter{dr. hab. Grzegorza Marcina Wójcika, prof. UMCS}
\degreedate{Lublin rok 2023}

\begin{document}

% This baselineskip gives sufficient line spacing for an examiner to easily markup the thesis with comments
\baselineskip=18pt plus1pt

% Set the number of sectioning levels that get number and appear in the contents
\setcounter{secnumdepth}{3}
\setcounter{tocdepth}{3}

% Add a title page
\maketitle
% Insert a blank page after the title page
\cleardoublepage

% Add an abstract in polish
\begin{abstract}
	Abstrakt w języku polskim
\end{abstract}

% Add an abstract in english
\include{abstract-en}

% Start roman page numbering
\begin{romanpages}
	% Generate and include a table of contents
	\tableofcontents
	% Create a group for list of figures and list of tables to fit both on one page
	\begingroup
	\let\clearpage\relax
	% Generate and include a list of figures
	\listoffigures
	% Generate and include a list of tables
	\listoftables
	\endgroup
% End roman page numbering
\end{romanpages}

% Include the introduction
\chapter*{Wstęp}
\addcontentsline{toc}{chapter}{Wstęp}
\label{ch:introduction}

Treść wstępu pracy

% Include the first chapter
\chapter{Tytuł rozdziału 1}

Rozdział pierwszy

\section{Sekcja 1 rozdziału 1}

Pierwsza sekcja rozdziału pierwszego

\section{Sekcja 2 rozdziału 1}

Druga sekcja rozdziału pierwszego

% Include the second chapter
\chapter{Tytuł rozdziału 2}

Rozdział drugi

\section{Sekcja 1 rozdziału 2}

Pierwsza sekcja rozdziału drugiego

\section{Sekcja 2 rozdziału 2}

Druga sekcja rozdziału drugiego

% Include the third chapter
\chapter{Tytuł rozdziału 3}

Rozdział trzeci

\section{Sekcja 1 rozdziału 3}

Pierwsza sekcja rozdziału trzeciego

\section{Sekcja 2 rozdziału 3}

Druga sekcja rozdziału trzeciego


% Add the Bibliography to the contents page
\addcontentsline{toc}{chapter}{Bibliografia}
% Use a bibtex bibliography file refs.bib
\bibliography{refs}
% Use the plain bibliography style ordered by citation
\bibliographystyle{unsrt}

\end{document}
