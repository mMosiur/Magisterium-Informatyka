% Two-sided document
\documentclass[12pt, twoside]{ociamthesis}
\usepackage{helvet}
\usepackage{amssymb}
\usepackage[utf8]{inputenc}
\usepackage[T1]{polski}
\usepackage{url}
\usepackage[hidelinks]{hyperref}
\usepackage[table,xcdraw]{xcolor}
\usepackage[leftbars]{changebar}
\usepackage{caption}
\hypersetup{
	pdftitle={Detekcja choroby Alzheimera i stadium demencji z użyciem narzędzi uczenia maszynowego w środowisku .NET},
	pdfsubject={Praca magisterska},
	pdfauthor={Mateusz Piotr Moruś},
	pdfkeywords={Choroba Alzheimera, demencja, uczenie maszynowe, .NET, dotnet, CSharp}
}
\usepackage{graphicx}
\usepackage{listings}
\usepackage{xcolor}

\definecolor{codegreen}{rgb}{0,0.6,0}
\definecolor{codegray}{rgb}{0.5,0.5,0.5}
\definecolor{backcolour}{rgb}{0.95,0.95,0.92}

\lstdefinestyle{mystyle}{
    backgroundcolor=\color{backcolour},
    commentstyle=\color{codegreen},
    keywordstyle=\color{magenta},
    numberstyle=\tiny\color{codegray},
    stringstyle=\color{codegreen},
    basicstyle=\ttfamily\footnotesize,
    breakatwhitespace=false,
    breaklines=true,
    captionpos=b,
    keepspaces=true,
    numbers=left,
    numbersep=5pt,
    showspaces=false,
    showstringspaces=false,
    showtabs=false,
    tabsize=4
}

\lstset{style=mystyle}


\setlength{\changebarsep}{-5em}

\graphicspath{ {./Images/} }

\captionsetup[figure]{font=small}

% Make the numbering of figures continuous
\counterwithout{figure}{chapter}
% Make the numbering of tables continuous
\counterwithout{table}{chapter}

\title{Detekcja choroby Alzheimera i stadium demencji z użyciem narzędzi uczenia maszynowego w środowisku .NET}
\engtitle{Detection of Alzheimer's disease and the stage of dementia using machine learning tools in the .NET environment}
\author{Mateusz Piotr Moruś}
\albumnumber{292540}
\university{UNIWERSYTET MARII CURIE-SKŁODOWSKIEJ W~LUBLINIE}
\college{Wydział Matematyki, Fizyki i~Informatyki}
\field{Informatyka}
\speciality{Deweloperska (programistyczna)}
\submittedtext{Praca magisterska}
\department{Katedrze Neuroinformatyki i~Inżynierii Biomedycznej}
\promoter{dr. hab. Grzegorza Marcina Wójcika, prof. UMCS}
\degreedate{Lublin rok 2023}

\begin{document}

% This baselineskip gives sufficient line spacing for an examiner to easily markup the thesis with comments
\baselineskip=18pt plus1pt

% Set the number of sectioning levels that get number and appear in the contents
\setcounter{secnumdepth}{3}
\setcounter{tocdepth}{3}

% Add a title page
\maketitle
% Insert a blank page after the title page
\cleardoublepage


\begin{abstract}

Abstrakt w języku polskim XD

\end{abstract}

\begin{abstract-en}

Abstract in english

\end{abstract-en}


% Start roman page numbering
\begin{romanpages}
	% Generate and include a table of contents
	\tableofcontents
	% Create a group for list of figures and list of tables to fit both on one page
	\begingroup
	\let\clearpage\relax
	% Generate and include a list of figures
	\listoffigures
	% Generate and include a list of tables
	\listoftables
	\endgroup
% End roman page numbering
\end{romanpages}

\chapter*{Wstęp}
\addcontentsline{toc}{chapter}{Wstęp}

Wśród wyzwań współczesnej medycyny i~informatyki, niezmiernie istotnym tematem stała się detekcja chorób neurodegeneracyjnych, zwłaszcza takiej jak choroba Alzheimera czy szerzej pojęta demencja.
Rozwój technologii i~zdolność do analizy ogromnych ilości danych stworzyły nowe możliwości w~dziedzinie diagnostyki i~przewidywania tych złożonych schorzeń.
Przy jednoczesnym użyciu potencjału narzędzi uczenia maszynowego oraz środowiska \emph{.NET}, otwierają się przed nami perspektywy badawcze, które mogą doprowadzić do istotnych postępów w~zrozumieniu oraz wczesnym wykrywaniu choroby Alzheimera.

Choroba Alzheimera, stanowiąca formę demencji, jest poważnym wyzwaniem dla opieki zdrowotnej na całym świecie.
Jej postępujący charakter i~wpływ na jakość życia pacjentów oraz ich bliskich stanowią bodziec do poszukiwania nowych narzędzi diagnostycznych.
Jest ona też najczęstszą przyczyną demencji, stanowiąc około $60$-$80$\% wszystkich stwierdzonych przypadków \cite{what-is-alzheimers:2023}.
Przeszłość medycyny skupiła się na metodach klinicznych i~obrazowaniu mózgu, jednak ostatnie lata przyniosły rewolucję w~zastosowaniu uczenia maszynowego do analizy medycznych danych.
W szczególności, techniki głębokiego uczenia maszynowego, które wykorzystują sieci neuronowe, stały się obiecującym narzędziem do diagnozowania chorób neurodegeneracyjnych.
W ostatnich latach, wiele badań skupiło się na wykorzystaniu obrazowania mózgu do detekcji choroby Alzheimera.

Z oczywistych względów najczęściej wykorzystywane środowiska i~narzędzia programistyczne używane w~celu uczenia modeli przeznaczonych do diagnostyki to te same, które cieszą się największą popularnością uniwersalnie w~dziedzinie inżynierii danych i~uczenia maszynowego -- w~głównej mierze języki Python i~R.

Jednakże w~samych zastosowaniach i~oprogramowaniach w~medycynie najczęściej spotyka się inne języki inne platformy, w~dominującej części jest to Java oraz na drugiej pozycji \emph{.NET} i~język C\#.
Java jest popularna zarówno w środowiskach akademickich jak i~medycznych, a~co za tym idzie podejmowanych jest wiele prób prowadzenia rozwojowych badań i~testów wykorzystania uczenia maszynowego w~celach predykcyjnych i~diagnostycznych \cite{soman2005classification, nithya2017predictive, gobbel2014development, godara2016evaluation}.
.NET natomiast jest raczej negatywnie odbierany w~środowisku akademickim a~popularność natomiast osiągnął głównie w~sferze rozwiązań komercyjnych -- w~tym również w~wielu systemat medycznych i~szpitalnych.
Więc mimo potencjału integracji rozwiązań uczenia maszynowego z~systemami medycznymi, wciąż uczenie maszynowe w~\emph{.NET} nie jest tematem często spotykanym.
Dlatego też postanowiłem sprawdzić, jakie możliwości daje nam platforma .NET w~kontekście uczenia maszynowego w~zastosowaniu do diagnostyki choroby Alzheimera.

Przyjrzenie się technologii głębokiego uczenia maszynowego oraz ich implementacji w~środowisku .NET jawi się jako kluczowe.
Platforma \emph{.NET}, z~jej wszechstronnymi językami programowania oraz narzędziami, stanowi wyjątkowe miejsce gdzie, w~różnorodnych rozwiązaniach wdrożyć można dodatkowe atuty płynące z~uczenia maszynowego.
Skoncentrowanie się na narzędziu \emph{ML.NET} oraz bibliotece \emph{TensorFlow.NET} umożliwia nam zbadanie najpopularniejszych i~najbardziej rozwiniętych podejść do uczenia modeli diagnostycznych w~kontekście choroby Alzheimera.

Celem niniejszej pracy jest nie tylko zgłębienie wiedzy na temat choroby Alzheimera oraz jej objawów, ale przede wszystkim wykorzystanie osiągnięć dziedziny uczenia maszynowego w~celu opracowania skutecznych modeli diagnostycznych.
Skonfrontowanie różnych podejść i~narzędzi pozwoli na lepsze zrozumienie procesu detekcji tej choroby oraz wydobycie potencjalnych strategii poprawy skuteczności i~efektywności diagnozowania.

W dalszych sekcjach pracy, zagłębimy się w~przeszłość i~teraźniejszość choroby Alzheimera, omówimy kluczowe metody diagnostyki oraz zaprezentujemy narzędzia i~technologie, które wykorzystamy do naszych badań porównawczych a~także ich wyniki i~wnioski.

\chapter{Przegląd piśmiennictwa}

Choroba Alzheimera nadal nie jest w pełni zrozumiana, jednak dzięki badaniom zdobyto już wiele istotnych informacji na jej temat.
Część z nich opiera się wyłącznie na korelacjach i potwierdzonych powiązaniach z pominięciem przyczyn i konkretnych mechanizmów nimi rządzących, lecz nadal daje nam to wystarczający obraz by przedstawić kompleksowo opis choroby.

\section{Historia, definicja i objawy choroby Alzheimera}

\subsection{Wczesna historia}

Istnienie demencji było znane ludzkości od bardzo dawna.
Pierwsze wzmianki na ten temat pojawiały się już w czasach starożytnej Grecji, a informacje dotyczące jej objawów można odnaleźć nawet wcześniej, w starożytnym Egipcie \cite{boller1998history}.

Natomiast choroba Alzheimera została po raz pierwszy opisana przez niemieckiego lekarza Aloisa Alzheimera, od którego nazwiska została później nazwana.
W roku 1901 neuropatolog zajął się przypadkiem Auguste Deter, pacjentki przyjętej do szpitala psychiatrycznego.
Od jej 51. roku życia zaczęły się stopniowe zmiany osobowości, które następnie przerodziły się w postępujące zaburzenia kognitywne, a ostatecznie w całkowitą apatię \cite{cipriani2011alzheimer}.
W momencie, gdy Alzheimer rozpoczął badania nad nią, pacjentka miała już prawie pięcioletnią historię narastających halucynacji, urojeń, apraksji, zaburzeń pamięci, mowy, społecznych i behawioralnych.

Po śmierci pacjentki lekarz przeprowadził badania mózgu zmarłej, w trakcie których znalazł zmiany i cechy dziś przypisywane omawianej chorobie.
Alzheimer przedstawił swoje kliniczne i patologiczne wyniki na konferencji południowo-zachodnich niemieckich psychiatrów w 1906 roku, następnie opublikował w formie pisemnej w czasopismach medycznych poświęconych psychiatrii \cite{alzheimer1906uber}.

W tym czasie jeszcze sam Alois Alzheimer nie był pewien, czy przypadek Auguste Deter reprezentował nowy zespół chorobowy.
Uważał, że opisał bardzo rzadką przypadłość i najprawdopodobniej wierzył, że była to po prostu szczególna forma demencji starczej \cite{cipriani2011alzheimer}.
Dopiero kilka lat później, po raz pierwszy użyty został termin ``choroba Alzheimera'' (\emph{ang. ``Alzheimer's disease'', często po prostu ``AD''}) w wydanym podręczniku psychiatrii Emila Kraeplina \cite{kraepelin1910psychiatrie}.

Jeszcze w pierwszej połowie XX wieku publikacje podkreślały brak zależności przebiegu choroby Alzheimera od wieku pacjentów.
Pojawiła się również istotna kliniczna debata wokół pytania czy demencja ``przedstarcza'' oraz demencja ``starcza'' (\emph{ang. kolejno ``presenile'' oraz ``senile''}) są tym samym zaburzeniem \cite{jellinger2006alzheimer}.
W połowie wieku pojawiły się pierwsze postulaty mówiące, że (przedstarcze) \emph{AD} oraz demencja starcza są całkowicie odrębnymi jednostkami chorobowymi (choć nowsze badania dojść jednoznacznie pokazują że tak nie jest).

W latach 60 XX wieku rozpoczęła się nowa era badań bazująca na nowym, bardzo ważnym narzędziu badawczym -- mikroskopii elektronowej.
Dzięki niej możliwe było zbadanie struktury komórek nerwowych i ich połączeń, a także zbadanie struktury i składu splotów nerwowych znajdujących się w mózgu.
Takich opisów jako jeden z pierwszych dostarczył w 1963 roku Kidd, który opisał zmiany w strukturze splątków neurofibrylarnych oraz blaszek amyloidowych \cite{kidd1963paired}.

Od tego czasu choroba progresywnie zyskiwała coraz większą uwagę ze strony środowiska naukowego, a także społeczeństwa.
W 1974 roku rząd Stanów Zjednoczonych Ameryki powołał do istnienia Narodowy Instytut ds. Starzenia się (\emph{ang. National Institute on Aging, NIA}), który od tego czasu finansuje badania nad chorobą Alzheimera \cite{marx1974aging}.
W 1980 roku powstało ``Stowarzyszenie Choroby Alzheimera i Pokrewnych Zaburzeń'' (\emph{ang. Alzheimer's Disease and Related Disorders Association, ADRDA}), organizacja non-profit, która dziś znana jest jako ``Stowarzyszenie Choroby Alzheimera'' (\emph{ang. Alzheimer's Association, AA}).
Wraz z rosnącą świadomością społeczną wzrastało także finansowanie, dzięki czemu choroba jeszcze 30 lat temu uznawana za całkiem nieuleczalną i nie do powstrzymania czy spowolnienia, dziś jest jedną z najbardziej intensywnie badanych chorób neurodegeneracyjnych z prospektami na walkę z nią w niedalekiej przyszłości.

Mimo tych wielu obecnych badań, nadal pozostaje wiele czynników wciąż niepoznanych i nie ma jednoznacznej odpowiedzi na pytanie co jest przyczyną choroby Alzheimera.
Wciąż brakuje również skutecznych metod leczenia, które mogłyby zatrzymać jej postęp lub spowolnić go w znaczący sposób.

\subsection{Współczesna definicja choroby Alzheimera}

W celu lepszego zrozumienia choroby Alzheimera, warto najpierw przyjrzeć się dwóm definicjom -- czym jest demencja oraz czym jest choroba Alzheimera -- jako, że terminy te często w świadomości publicznej są utożsamiane, co nie jest prawdą, choć pojawiają się miedzy nimi istotne powiązania.

Demencja (inaczej \emph{otępienie}) to znacznie ogólniejszy termin określający utratę pamięci, zdolności językowych, umiejętności rozwiązywania problemów i utrudnienie innych czynności poznawczych, które są na tyle poważne, że przeszkadzają w codziennym życiu.
Nie odnosi się w żadnym wypadku do naturalnego procesu starzenia się, i nie należy drobnych problemów z pamięcią czy zapominalstwa z nią mylić.
Demencja jest też objawem, a nie chorobą, i może być spowodowana przez wiele różnych chorób i zaburzeń.

Choroba Alzheimera jest jedną z wielu chorób, które mogą prowadzić do demencji.
Odpowiada ona za między 60\% a 80\% przypadków demencji, co czyni ją najczęstszą przyczyną otępienia \cite{what-is-alzheimers:2023}.
Obecnie klinicyści używają terminu \emph{AD} -- choroba Alzheimera -- w odniesieniu do zespołu chorobowego, który objawia się charakterystycznym postępującym zaburzeniem amnezyjnym z późniejszym pojawianiem się innych zmian poznawczych, behawioralnych i neuropsychiatrycznych, które upośledzają funkcje społeczne i czynności życia codziennego \cite{cummings2004alzheimer}.

Mimo, że choroba Alzheimera jest najczęstszą przyczyną demencji, to jednak nie jest z nią tożsama.
Drugą najczęstszą przyczyną jest demencja naczyniowa, która jest spowodowana uszkodzeniem naczyń krwionośnych w mózgu, najczęściej w wyniku udaru mózgu.


\subsection{Objawy choroby Alzheimera i stadium demencji}

Objawy choroby Alzheimera są bardzo zróżnicowane i zależą od stadium choroby.
Dodatkowo wczesne objawy często są mylone z naturalnym procesem starzenia się, co utrudnia wczesną diagnozę.

Ogólnie rzecz biorąc, objawy choroby Alzheimera można podzielić na 3 główne etapy powiązane ze stadium choroby \cite{alzheimers-symptoms:2021}:

\begin{itemize}

  \item Wczesne objawy

        W początkach rozwijającej się choroby głównym symptomem są zaniki pamięci.
        Można tutaj wymienić między innymi zapominanie o niedawnych rozmowach lub wydarzeniach, gubienie przedmiotów, zapominać nazw miejsc i rzeczy, powtarzanie się, trudności w znalezieniu właściwego słowa czy podejmowaniu decyzji.
        Łatwo zauważyć, że są to objawy, które mogą pojawić się w mniejszym stopniu u każdego człowieka, w tym w pełni zdrowego, co znacznie utrudnia wczesną diagnozę -- nasilenia się tych objawów są często mylone z naturalnym procesem starzenia się.

  \item Objawy w średnim stadium

        Narastają przypadku dezorientacji - na przykład gubienie się, błądzenie czy brak świadomości pory dnia.
        Chory może powtarzać czynności, mieć urojenia, odczuwać paranoję, zaburzony jest sen.
        AD może również wpływać na psychikę, powodując częste zachwiania nastroju, niepokój, frustrację czy depresję \cite{li2014behavioral}.
        Niebezpieczne mogą być także problemy z percepcją -- ocena odległości czy uświadamiane widzenie lub słyszenie.

        Wraz z rozwojem choroby Alzheimera, problemy z pamięcią będą się pogarszać.
        Pogłębiają się problemy z zapamiętywaniem imion, nawet tych osób, które zna się od dawna.
        Mogą pojawiać się także trudności w rozpoznawaniu miejsc i osób.
        Ciężkim, możliwym objawem pojawiającym się w średnim stadium jest także afazja, czyli zaburzenia mowy i języka.

        Na tym etapie osoba cierpiąca na chorobę Alzheimera zazwyczaj potrzebuje wsparcia w codziennym życiu.

  \item Późniejsze objawy

        W późniejszych stadiach choroby Alzheimera objawy stają się coraz bardziej nasilone i mogą być w znacznym stopniu wpływać na osobę chorą, a także jej opiekunów, przyjaciół i rodziny.

        Mogą pojawiać się halucynacje i urojenia, zanikające lub nasilające się wraz z postępem choroby.
        Czasami osoby cierpiące na chorobę Alzheimera mogą być agresywne, wymagające i podejrzliwe w stosunku do otoczenia.

        W miarę postępu choroby Alzheimera może również rozwinąć się szereg innych objawów, takich jak dysfagia (trudności z połykaniem), nietrzymanie moczu czy stolca, problemy z poruszaniem się bez pomocy, fizyczna utrata masy ciała, a także stopniowa i często szybka utrata zdolności poznawczych -- w tym mowy \cite{joe2019cognitive}.
        Znacznie pogłębiają się problemy z krótko- i długoterminową.

        W ciężkich stadiach choroby Alzheimera ludzie mogą potrzebować całodobowej opieki oraz pomocy w jedzeniu, poruszaniu się i pielęgnacji osobistej.

\end{itemize}

\section{Diagnoza choroby Alzheimera}

Mimo, że choroba Alzheimera nadal nie jest w pełni zrozumiana i zbadana oraz wciąż nie ma skutecznych metod jej leczenia, to obszar diagnozy i wczesnego wykrywania choroby jest bardzo dobrze rozwinięty.
Powstało wiele metod i narzędzi diagnostycznych, które pozwalają na wczesne wykrycie choroby.

\subsection{Metody kliniczne diagnostyki choroby Alzheimera}

Między rokiem 1960 a 1985 wprowadzono kilka różnych kategorii obiektywnych klinicznie narzędzi diagnostyki choroby Alzheimera, wiele z nich potwierdzonych w autopsji \cite{khachaturian2006diagnosis}.
Były to między innymi:

\begin{itemize}

  \item \emph{Mini-Mental State Examination} (w skrócie \emph{MMSE}, potocznie mini-mental) -- jest to zaprojektowany w 1975 roku test oceny demencji oraz jej stopnia w formie szybkiego badania, najczęściej krótszego niż 5 minut.
        Mini-mental jest narzędziem przesiewowym (ang. \emph{screening}), to jest -- nie jest dokładne i jednoznaczne, ale jego negatywny wynik daje podstawy przypuszczać prawdopodobną utratę zdolności kognitywnych i wszcząć
        głębsze, kliniczne badania w celu potwierdzenia wstępnej diagnozy.

        W wyniku tego badania uzyskuje się miedzy 0 a 30 punktów, gdzie wraz z malejącą zdobytą ilością rośnie domniemany stopień demencji.
        27 punktów oraz powyżej oznacza wynik prawidłowy, 24 i w górę to niedemenecyjne zaburzenia poznawcze, 19 wzwyż sugerują demencję łagodną, od 11 do 18 demencję średniego stopnia, a poniżej 10 demencję głęboką.

        Niedawne badania nad skutecznością MMSE na przestrzeni różnych kategorii stwierdzały jej czułość (sensitivity) między 23\% a 76\%, a jej specyficzność (specificity) między 40\% a 90\% \cite{arevalo2015mini}.
        Nie jest to więc wynik wysoki, jednakże wciąż jest to istotne narzędzie używane w praktyce klinicznej, głównie ze względu na jego prostotę i szybkość.

  \item \emph{Short Blessed Test} (w skrócie \emph{SBT}) -- badanie opublikowane do użytku w 1983 roku skupiające się na orientacji, pamięci i koncentracji \cite{katzman1983validation}.
        Według samych autorów przeznaczone do wykrywania wczesnych zaburzeń poznawczych, nie samej diagnozy choroby Alzheimera.
        Wszelkie wyniki sugerujące demencję wymagają dalszych badań.

  \item \emph{Clinician's Interview-Based Impression of Change plus caregiver input} (w skrócie \emph{CIBIC+}) -- zaproponowane i wymagane przez amerykańską Agencję Leków i Żywności (ang. \emph{FDA}, \emph{Food and Drug Administration}) podejście do przeprowadzania badań nad lekami na chorobę Alzheimera i demencję, który ma pozwalać na uzyskiwanie mierzalnych klinicznie danych potrzebnych do oceny skuteczności badanych środków \cite{joffres2000qualitative}.
        Najczęściej praktykowanym testem implementującym założenia CIBIC+ jest \emph{Alzheimer's Disease Cooperative Study -- Clinical Global Impression of Change} (w skrócie \emph{ADCS-CGIC}) z roku 1990.

  \item \emph{Clinical Dementia Rating} (w skrócie \emph{CDR}) -- skala numeryczna używana do ilościowego określenia nasilenia się objawów demencyjnych i skategoryzowania choroby w określonym "stadium".
        Wynik w tej skali otrzymuje się jako rezultat przeprowadzenia wywiadu z pacjentem opracowanego w 1993 roku przez Morrisa i jego współpracowników, lekarzy z Washington University School of Medicine \cite{morris1993clinical}.

        CDR uważany jest za test zdolny do rozpoznania bardzo łagodnych upośledzeń, co przeważnie nie było możliwe w poprzednich testach klinicznych dotyczących choroby Alzheimera.
        Ma on jednak słabe strony, w tym najważniejszy -- długi czas potrzebny na przeprowadzenie całościowego wywiadu, i co za tym idzie jest względnie niezdolny do uchwycenia zmian w czasie.
        Dodatkowo polega on na subiektywnej ocenie lekarza, co może prowadzić do błędów.

  \item \emph{Alzheimer's Disease Assessment Scale -- Cognitive Subscale} (w skrócie \emph{ADAS--Cog}) -- test kliniczny, który mierzy zmiany w funkcjonowaniu poznawczym pacjenta i nasilenie objawów demencji.
        Jest połową większego testu, \emph{Alzheimer's Disease Assessment Scale} (w skrócie \emph{ADAS}), który składa się z dwóch części: \emph{ADAS--Cog} oraz \emph{ADAS--Noncog}.
        ADAS--Cog zawiera 11 pytań, które oceniają pamięć, orientację, rozumienie, umiejętności wykonawcze i język.
        Do dnia dzisiejszego jest on na szeroką skalę używany w badaniach klinicznych nad lekami na demencję czy chorobę Alzheimera i uważany za najlepszy standard  ocenie leczenia przeciw otępieniu \cite{connor2008administration}.

\end{itemize}

Większość tego typu badań ma charakter niekonkluzywny i nie są w stanie jednoznacznie potwierdzić czy wykluczyć chorobę Alzheimera.
Wymagają one dalszych badań, które mogą być zarówno kliniczne, jak i laboratoryjne.
Część z nich jednak -- w szczególności ostatni z wymienionych, \emph{ADAS--Cog} -- są na tyle dokładne, że do dziś są bardzo szeroko używane w badaniach klinicznych nad chorobą Alzheimera.
A z kolei wszystkie są wystarczająco skuteczne, by być podstawą do podjęcia decyzji i przeprowadzeniu znacznie dokładniejszych badań laboratoryjnych.
Takimi dokładnymi badaniami biologicznymi mogą być pośmiertne badania neuropatologiczne, ale także możliwe jest w trakcie życia pacjenta badanie biomarkerów wskazujących na chorobę Alzheimera \cite{mantzavinos2017biomarkers}.

Nowoczesne uniwersalne narzędzia diagnostyczne są również bardzo przydatne w badaniach nad chorobą Alzheimera, w szczególności możliwość dokładnego obrazowania mózgu.

\subsection{Metody obrazowania mózgu w diagnostyce choroby Alzheimera}

Sugestie dotyczące przydatności i skuteczności nowych technik obrazowania mózgu w diagnostyce choroby Alzheimera pojawiły się już w latach 80 XX wieku \cite{mcgeer1986brain}.
Testowano wykorzystanie tomografii komputerowej (\emph{CT}), rezonansu magnetycznego (\emph{MRI}), pozytonowej tomografii emisyjnej (\emph{PET}) oraz pojedynczej emisji fotonów (\emph{SP}).
Autorzy wskazywali jednak, że skany tomografii komputerowej i rezonansu magnetycznego ujawniają uogólniony proces zanikowy kory mózgowej, który nasila się wraz z wiekiem i mimo, że proces zanikowy jest bardziej nasilony w chorobie Alzheimera, to jednak nie może być stosowany jako metoda diagnostyczna, ponieważ nie występuje w każdym przypadku.
Dodali też, że zmian nie można przewidzieć z użyciem tych dwóch wspomnianych metod.

Nowsze prace jednak nie są tak rygorystyczne w stosunku do metod obrazowania mózgu i zaznaczają, że zarówno pierwotnie używana tomografia komputerowa, a także następnie wprowadzony rezonans magnetyczny były i są z powodzeniem używane w diagnostyce choroby Alzheimera \cite{johnson2012brain}.
Co więcej, funkcjonalny rezonans magnetyczny (\emph{fMRI}) czy pozytonowa tomografia emisyjna (\emph{PET}) potrafiły wykazać charakterystyczne zmiany w mózgach pacjentów z \emph{AD} w stadiach nawet przedobjawowych,
Żadna z metod obrazowania nie może służyć wszystkim celom, ponieważ każda z nich ma unikalne mocne i słabe strony.

Bardzo świeże prace z 2022 roku opisują, że obecnie neuroobrazowanie znajduje się w ścisłej czołówce metod pomocnych w diagnozowaniu choroby Alzheimera (\emph{AD}), a także innych wszelkich innych rodzajów demencji, takich jak otępienie czołowo-skroniowe, otępienie naczyniowe i otępienie z ciałami Lewy'ego (\emph{DLB}) \cite{scheltens2022imaging}.
Niedawno sformułowane przez Dubois i współpracowników kryteria badawcze dla wczesnej choroby Alzheimera wyraźnie wymieniają obrazowanie rezonansu magnetycznego (\emph{MRI}) i pozytonową tomografię emisyjną (\emph{PET}) dla \emph{AD} i są przykładem rozwoju nowego procesu diagnostycznego.
Autorzy sugerują, że wkrótce neuroobrazowanie zostanie włączone do kryteriów diagnostycznych dla różnych rodzajów demencji.

\subsection{Nowoczesne metody detekcji choroby Alzheimera z użyciem uczenia maszynowego}

Diagnostyka choroby Alzheimera miała duże skoki rozwojowe wraz z powstawaniem nowych technologii i metod badawczych.
Już w połowie lat 90 XX wieku pojawiły się pierwsze prace naukowe próbujące wykorzystać rozwijającą się dziedzinę uczenia maszynowego przy detekcji demencji.
Te pierwsze próby opierały się jednak nie na obrazach, ale na analizie danych klinicznych i rezultatów testów psychometrycznych \cite{datta1996applying}.
Rezultaty pokazały, że już te proste podejścia mają skuteczność nieco lepszą niż kwestionariusze zaprojektowane do ręcznej analizy zebranych danych.

Rosnąca popularność neuroobrazowania w diagnostyce choroby Alzheimera oraz dalszy rozwój technik uczenia maszynowego i sztucznej inteligencji doprowadziły do powstania wielu prac naukowych, które łączą te dwa obszary.

W 2007 roku podjęto próbę analizy kształtu hipokampu u pacjentów z chorobą Alzheimera z użyciem technik uczenia maszynowego \cite{li2007hippocampal}.
W wyniku uznano, że subtelne i przestrzennie złożone wzorce deformacji hipokampa pomiędzy pacjentami posiadającymi AD a zdrowymi osobami kontrolnymi można wykryć za pomocą metod uczenia maszynowego.

Dalsze badania sprawdzały skuteczność technik klasyfikacji w uczeniu maszynowym obrazów rezonansu magnetycznego w celu detekcji choroby Alzheimera \cite{herrera2013classification}.
Klasyfikacja miała być binarna, czyli obrazy były przydzielane do kategorii osób zdrowych lub kategorii osób z chorobą Alzheimera, bez podziału na stadium choroby.
Powstały model miał służyć jako pomoc we wczesnym sygnalizowaniu potencjalnych wskaźników choroby Alzheimera.

Nowe prace analizowały wykorzystywane do celów diagnostyki metody uczenia maszynowego na przestrzeni lat 2005 do 2019 \cite{tanveer2020machine}.
Zamknięto pojawiające się techniki w 3 kategoriach: maszyny wektorów nośnych (ang. \emph{Support Vector Machine}, \emph{SVM}), sztuczne sieci neuronowe (ang. \emph{Artificial Neural Network}, \emph{ANN}) oraz uczenie głębokie (ang. \emph{Deep Learning}, \emph{DL}).
Jak się okazuje, najwięcej prac naukowych używało technik maszyn wektorów nośnych \emph{SVM} ze względu na ich powtarzalność i solidność wyników, nie cierpią one bowiem na problemy związane z lokalnymi minimami.
Jednak sieci neuronowe są bardziej wszechstronne jeśli chodzi o przyrostowe uczenie się i modelowanie danych sekwencyjnych.
Najbardziej obiecująca jednak była technika uczenia głębokiego, dzięki której można modelować bardzo złożone dane z dużą dokładnością.

W tę właśnie stronę poszli autorzy późniejszej pracy, którzy używając obrazów rezonansu magnetycznego MRI wyszkolili głęboki model dla detekcji choroby Alzheimera \cite{ebrahimi2021deep}.
Wykorzystali oni konwolucyjne głębokie sieci neuronowe i różne rodzaje sieci rekurencyjnych bazowanych na wstępnie wytrenowanym modelu ResNet-18.
Otrzymali bardzo obiecujące wyniki i model, który osiągnął maksymalną wydajność klasyfikacji z dokładnością, czułością i swoistością na poziomie około 92\%.

\section{Uczenie maszynowe i techniki klasyfikacji}

W celu lepszego zrozumienia dlaczego i w jaki sposób uczenie maszynowe może być bardzo przydatnym narzędziem w diagnostyce choroby Alzheimera, a także całym ogólniej w obszarze medycyny i diagnostyki, warto przyjrzeć się na czym polega i jakie są jego podstawowe techniki.

\subsection{Generalna definicja uczenia maszynowego}

Uczenie maszynowe (często w skrócie \emph{ML}, z ang. \emph{Machine Learning}) jest pojęciem bardzo szerokim i z natury interdyscyplinarnym, natomiast w najpopularniejszym użyciu odnosi się do kategorii sztucznej inteligencji, która pozwala komputerom ``myśleć'' i uczyć się we własnym zakresie.
Sam termin został po raz pierwszy użyty przez Arthura Samuela w 1959 roku, który to zdefiniował uczenie maszynowe jako dziedzinę studiów, która daje komputerom możliwość uczenia się bez ich bezpośredniego programowania \cite{samuel1959some}.

Taka definicja uczenia maszynowego, mimo, że ogranicza się do jednej z wielu dziedzin potencjalnie zaliczanych do tej kategorii, jest bardzo ogólna i nie precyzuje dokładnie co to znaczy ``uczyć się'' ani jak takie uczenie się miałoby być manifestowane.
Jest to zabieg celowy, ponieważ -- w przeciwieństwie do tego, co pozwalałyby sądzić najpopularniejsze obecnie metody uczenia maszynowego -- nie jest to jedynie zbiór badań nad sieciami neuronowymi i systemami ekspertowymi.
Wspomniana definicja jednak jest zbyt ogólna, by móc powiedzieć o niej coś więcej bez bazowania na przykładowych technikach i algorytmach.

Aby sformalizować pewne uniwersalne właściwości technik uczenia maszynowego, zaproponowany został podstawowy model uczenia maszynowego.
W nim uczenie maszynowe zostało zaprezentowane jako proces składający się z czterech głównych elementów: środowiska, uczenia, repozytorium oraz wykonania \cite{wang2009brief}.
Wygląd takiego modelu został przedstawiony na \hyperref[fig:basic-model-of-machine-learning]{rysunku \ref*{fig:basic-model-of-machine-learning}}.

\begin{figure}[ht]
  \includegraphics[width=\textwidth]{basic-model-of-machine-learning}
  \caption[Podstawowy model uczenia maszynowego]{Podstawowy model uczenia maszynowego zaadoptowany z pracy Hua Wang, Cuiqin Ma oraz Lijuan Zhou \cite{wang2009brief}. Bloki prostokątne oznaczają systemy a bloki procesowe (przypominające strzałki) oznaczają procesy.}
  \label{fig:basic-model-of-machine-learning}
\end{figure}

\begin{itemize}

  \item W modelu tym ``środowisko'' jest zewnętrznym zbiorem danych, które dostarczane są w pewnej bliżej nieokreślonej formie, reprezentuje źródła informacji zewnętrznych.
        Może to być na przykład zbiór obrazków przedstawiających odręcznie zapisane cyfry arabskie ręcznie oznaczonych rzeczywistą odpowiadającą im cyfrą przez człowieka.
        Mogą to być jednak także wszelkie inne dane pochodzenia zewnętrznego względem systemu uczenia maszynowego a które służą za podstawę do dalszego jego działania i kluczowego procesu uczenia się.


  \item Drugim czynnikiem tego modelu jest ``uczenie'', które jest procesem przetwarzającym informacje zewnętrzne na wiedzę.
        Terminy tutaj występujące są abstrakcją i mogą oznaczać wiele różnych rzeczy, w zależności od konkretnej omawianej techniki.
        Przykładowo w sztucznych sieciach neuronowych uczenie jest realizowane przez algorytm propagacji wstecznej błędu.
        Niezależnie od implementacji, wiedza jako szeroko pojęty rezultat uczenia jest trafia następnie do przechowania w repozytorium.

  \item Trzeci blok modelu, ``repozytorium'', jest miejscem przechowywania ogólnych zasad, które następnie kierują częścią działań wykonawczych systemu uczenia maszynowego.
        W przypadku sztucznych sieci neuronowych takim repozytorium jest sama struktura sieci oraz wagi połączeń między neuronami.
        Jest to szczególnie ciekawy przypadek, gdzie wiedza ta jest przechowywana w formie w zasadzie niedostępnej i niezrozumiałej dla człowieka, gdzie jej wydobycie jest bardzo trudnym zadaniem \cite{boger1997knowledge}.

  \item Ostatni blok modelu to ``wykonanie'', czyli proces, w którym system uczenia maszynowego wykorzystuje zgromadzoną w repozytorium wiedzę do podejmowania decyzji w kontekście problemu, który ma rozwiązać.
        Wynik działania procesu wykonawczego trafia później z powrotem do procesu uczenia, gdzie razem z dalszymi danymi pochodzącymi ze ``środowiska'' jest wykorzystywany do dalszego rozwoju wiedzy.

\end{itemize}

Cały proces zapętla się i pozwala na inkrementacyjne poprawianie wiedzy systemu uczenia maszynowego, aż do osiągnięcia pożądanej skuteczności procesu wykonawczego.
Tak przedstawiony model jest dużą abstrakcją pozostawiającą szczegóły implementacyjne i działania poszczególnych bloków poszczególnym technikom uczenia maszynowego, ale pozwala wyciąganie wniosków całościowych na temat uczenia maszynowego w pracach natury teoretycznej.

Jednak dążąc do lepszego poznania uczenia maszynowego od strony praktycznej, najlepiej jest przyjrzeć się kategoriom technik uczenia maszynowego, które są najczęściej wykorzystywane.

\subsection{Kategorie technik uczenia maszynowego}

Tradycyjnie wyróżnić można trzy paradygmaty ``uczenia'', na które dzieli się szerzej pojęte uczenie maszynowe.
Podział ten bazuje bezpośrednio na rozróżnieniu rodzajów sygnału -- danych dostarczanych do systemu uczenia maszynowego -- oraz informacji zwrotnej przekazywanej systemowi.
Te aspekty z kolei pośrednio wpływają na to, w jaki sposób w ogóle tego typu algorytm się zachowuje i jakie są jego możliwości oraz predyspozycje.
Kategorie te to:

\begin{itemize}

  \item \emph{Uczenie nadzorowane} (ang. \emph{supervised learning}) --
        algorytmy tego typu opierają się na analizie (czy przyswajaniu) zbiorów danych składających się zarówno z sygnału wejściowego, jak i opisanej już informacji zwrotnej.
        Dane wejściowe mogą być w dowolnej postaci, tekstowej, liczbowej, obrazkowej, dźwiękowej -- cokolwiek co może zostać zakodowane w sposób, w jaki algorytm reprezentuje dane.

        Co wyróżnia tę kategorię wśród pozostałych to fakt, że częścią danych przekazywanych do algorytmu jest także informacja na temat oczekiwanego wyjścia, zachowania się modelu.
        Przekazujemy algorytmowi to, jak powinien na podane dane wejściowe zareagować, a proces jego uczenia polega na dopasowaniu się do tych oczekiwań.

        Uczenie nadzorowane jest czasem nazywane także \emph{klasyfikacją} lub \emph{uczeniem indukcyjnym}, chociaż nie są to terminy ściśle równoważne.
        Pojawiają się jednak, ponieważ algorytmy tego typu przypominają rozumowanie indukcyjne, które polega na wyciąganiu ogólnych wniosków na podstawie konkretnych przykładów.
        Dodatkowo są szczególnie przydatne w zadaniach klasyfikacyjnych, w których radzą sobie bardzo dobrze \cite{hastie2009overview}.

        Do zadań klasyfikacyjnych wymagana jest możliwość podzielenia zbioru danych wejściowych na kategorie, gdzie każdy element przynależy do jednej z nich.
        Jednak uczenie nadzorowane także dobrze radzi sobie z problemami regresji, czyli przewidywania wartości liczbowych na podstawie danych wejściowych.
        Wtedy wyjściem nie jest dyskretna kategoria, ale liczba rzeczywista gdzie dozwolony jest pewien margines błędu -- algorytm ma dążyć jak najbliższej prawdziwej oczekiwanej wartości, jednak nie musi jej dokładnie osiągnąć.
        Przykładem może być system przewidywania cen nieruchomości na podstawie różnych cech mogących na nią wpływać.

        Systemu uczenia nadzorowanego nie potrafią jednak funkcjonować w izolacji -- potrzebują \emph{nadzoru}, zewnętrznej pomocy w postaci poprawnie opisanych lub skategoryzowanych danych wejściowych.
        W praktyce oznacza to, że uczenie nadzorowane jest zależne od człowieka, który musi dostarczyć algorytmowi dane wejściowe wraz z opisami, które często z angielskiego nazywane są \emph{labels}.
        Mimo skuteczności są więc ograniczane przez ilość danych, jaką jesteśmy w stanie dostarczyć, a także przez to, jak dobrze są one opisane.

        Popularnymi przykładami algorytmów uczenia nadzorowanego są między innymi \emph{regresja logistyczna} oraz \emph{k-najbliższych sąsiadów}.


  \item \emph{Uczenie nienadzorowane} (ang. \emph{unsupervised learning}) --
        systemy uczenia maszynowego tego typu różnią się od uczenia nadzorowanego w głównej mierze brakiem informacji zwrotnej.

        Otrzymują również zbiór danych wejściowych w dowolnej postaci, jednak nie są one opisane, nie wiemy odgórnie jak powinien zachować się algorytm na ich podstawie.

        Celem działania algorytmów uczenia nienadzorowanego jest znalezienie w nieopisanych danych wejściowych struktury i prawidłowości, na przykład ich zgrupowanie lub klasteryzacja (ang. \emph{clustering}) -- bezpośrednie wywnioskowanie bliżej nieokreślonych cech danych bez wsparcia z zewnątrz \cite{hastie2009unsupervised}.

        W uczeniu maszynowym tego typu w przeciwieństwie do uczenia nadzorowanego system zamiast reagować na informacje zwrotne powstałe poprzez porównanie opisu danych z wyjściem, algorytmy uczenia nienadzorowanego identyfikują podobieństwa w danych i reagują na podstawie obecności lub braku takich podobieństw w każdym nowym fragmencie zbioru wejściowego.

        Algorytmy uczenia nienadzorowanego są często wykorzystywane do analizy danych, w szczególności do wykrywania anomalii, czyli elementów zbioru danych, które wyróżniają się na tle pozostałych.

        Przykładem algorytmu uczenia nienadzorowanego jest algorytm centroidów, nazywany też algorytmem \emph{k-średnich} (ang. \emph{k-means}), który mimo swoich wad jest jednym z najpopularniejszych algorytmów klasteryzacji \cite{ahmed2020k}.
        Przez swoją nazwę jest często mylony lub błędnie utożsamiany z nadzorowanym podejściem k-najbliższych sąsiadów, jednak jest to zupełnie inny, niepowiązany algorytm.

  \item \emph{Uczenie przez wzmacnianie} (ang. \emph{reinforcement learning}) --
        podejście tego typu nieco różnią się od dwóch poprzednich, ponieważ nie są one zależne od predefiniowanego zbioru danych wejściowych, a od \emph{środowiska}, w którym algorytm się znajduje.
        Opierają się na ``nagradzaniu'' systemu za zachowania, które są pożądane i/lub ``karaniu'' za te niepożądane.

        Tym samym systemy takie są bliżej działania bazującego na modelu opartym na agentach (ang. \emph{agent-based model}, \emph{ABM}).
        Jest to model obliczeniowy służący do symulacji działań i interakcji autonomicznych agentów w celu zrozumienia zachowania systemu i tego, co rządzi jego wynikami.
        W omawianym kontekście oznacza to, że uczony przez wzmacnianie system jest traktowany jako ``agent'' w szeroko rozumianym środowisku, z którym wchodzi w różnego rodzaju interakcje.

        Ogólne założenia uczenia przez wzmacnianie przypominają nieco proces uczenia się, który widzimy u zwierząt i ludzi.
        Agent (system) jest umieszczany w środowisku, w którym może wykonywać pewne akcje.
        Wykonywane przez niego akcje prowadzą do pewnych rezultatów odbieranych przez niego jako nagrody lub kary.
        Otrzymywanie pozytywnych informacji zwrotnych -- nagród -- zachęca agenta do powtarzania akcji, które doprowadziły do ich otrzymania, a otrzymywanie negatywnych informacji zwrotnych -- kar -- zachęca do unikania prowadzących do nich czynności

        Agent skonstruowany jest tak, by chcieć maksymalizować otrzymywane nagrody, a minimalizować kary.
        Nazywane to jest funkcją nagrody (ang. \emph{reward function}).
        Dąży do tego sposobem, który w uproszczeniu można nazwać metodami prób i błędów.

        Przykładem algorytmów uczenia przez wzmacnianie jest metoda Monte Carlo.
        Jest to metoda statystyczna, która wykorzystuje powtarzalne losowania do rozwiązywania problemów, których rozwiązanie jest trudne lub niemożliwe do znalezienia w inny sposób.
        Agent ``próbkuje'' wykonywanie pewnej akcji i uśrednia uzyskane rezultaty -- wartości nagród/kar.
        Następnie uśredniony rezultat osiągnięty w trakcie trwania całości ``epizodu'', czyli pewnego konkretnego zdarzenia, jest wykorzystywany do zmiany zachowania agenta w przyszłości \cite{thrun2000reinforcement}.

\end{itemize}

W kontekście detekcji różnych jednostek chorobowych w medycynie, w tym choroby Alzheimera, najczęściej wykorzystywane są techniki uczenia nadzorowanego.
Pomimo, że uczenie nienadzorowane również dobrze sprawdza się w niektórych przypadkach \cite{raza2021tour} oraz uczenie przez wzmacniane także znajduje swoje zastosowania \cite{zhou2021deep}, to uczenie nadzorowane nadal sprawuje się najlepiej ze względu na naturę danych i problemów jakie najczęściej się w tym obszarze pojawiają.

Warto także dodać, że najnowsze popularne i bardzo obiecujące podejście -- tak zwane \emph{samo-nadzorowane uczenie maszynowe} (ang. \emph{self-supervised machine learning}) -- jest w istocie połączeniem technik uczenia nadzorowanego i nienadzorowanego i jest w stanie wykorzystać zalety obu tych podejść \cite{krishnan2022self}.

\subsection{Podstawowe algorytmy klasyfikacji}

Algorytmów wykorzystywanych do klasyfikacji jest bardzo wiele, warto jednak przyjrzeć się dwóm z prostszych i dobrze znanych.

\begin{itemize}

  \item \emph{Regresja logistyczna}

        Jest to algorytm nadzorowanego uczenia maszynowego wykorzystywany głównie w zadaniach klasyfikacyjnych, w których celem jest przewidywanie prawdopodobieństwa przynależności  -- danych wykorzystanych jako informacje wejściowe systemu -- do danej klasy.
        Jest ona określana jako regresja, ponieważ przyjmuje dane wyjściowe funkcji regresji liniowej jako dane wejściowe i wykorzystuje funkcję sigmoidalną do oszacowania prawdopodobieństwa dla danej klasy.

        Różnica między regresją liniową a regresją logistyczną polega na tym, że wynik regresji liniowej jest wartością ciągłą, która może być dowolna, podczas gdy regresja logistyczna przewiduje prawdopodobieństwo, że instancja należy do danej klasy lub nie.
        Wynikiem regresji liniowe jest pewna liczba rzeczywista w dowolnym przedziale i może być wykorzystywana na przykład we wspomnianym już zadaniu przewidywania cen nieruchomości.
        Natomiast regresja logistyczna jako wynik zwraca wartość z zakresu $0$ do $1$, która może być interpretowana jako prawdopodobieństwo przynależności do danej klasy.

        Z tego też powodu rodzaj klasyfikacji, do którego wykorzystywana jest regresja logistyczna nazywa się też \emph{klasyfikacją binarną}.
        W praktyce oznacza to, że algorytmy tego typu służą do rozróżnienia dokładnie dwóch klas.
        Mogą to być dwie klasy jednostkowe, na przykład czy dany obrazek przedstawia psa czy kota, ale bardzo często są to klasy oznaczające weryfikację ``pozytywną'' lub ``negatywną'' odnośnie posiadania pewnej cechy przez dane wejściowe lub możliwość ich przyporządkowania do jednej konkretnej klasy, na przykład czy dany pacjent jest chory czy nie (inaczej po prostu zdrowy), czy dana wiadomość e-mail jest spamem czy nie.

        Przy interpretacji pojedynczego zadania regresji liniowej jako binarnego przewidywania prawdopodobieństwa przynależności do danej klasy, algorytm ten nazywany jest także \emph{dwumianową regresją logistyczną} (ang. \emph{binomial logistic regression}).
        Algorytm ten może być rozszerzony jednak przez proste zwielokrotnienie takich klasyfikacji binarnych w celu rozróżnienia wielu klas.
        Wtedy nazywany jest \emph{wielomianą regresją logistyczną} (ang. \emph{multinomial logistic regression}) i potrafi kategoryzować dane wejściowe do dowolnej liczby klas, na przykład obrazki odręcznie zapisanych cyfr arabskich na 10 klas odpowiadających cyfrom od 0 do 9 \cite{palvanov2018comparisons}.

  \item \emph{K-najbliższych sąsiadów} (ang. \emph{k-nearest neighbors}, \emph{k-nn})

        Jest to drugi z bardzo dobrze znanych algorytmów klasyfikacyjnych godnych omówienia.
        Używany jest do klasyfikacji danych na podstawie podobieństwa do innych danych.
        Głównym jego założeniem jest reprezentacja przestrzeni problemowej w postaci wielowymiarowej przestrzeni euklidesowej, w której każdy punkt reprezentuje jeden z elementów zbioru danych wejściowych.
        Pozycja punktu jest określona na bazie jego cech przestawionych w postaci liczbowej, a każda z nich reprezentuje swój własny wymiar tejże przestrzeni \cite{kramer2013k}.

        Tak zareprezentowane dane wejściowe są następnie porównane między sobą przede wszystkim na podstawie odległości euklidesowej, czyli odległości między punktami w przestrzeni euklidesowej.
        Odległości te następnie służą jako podstawa do wyciągnięcia wniosków na temat kategorii obiektów.

        Pod uwagę brane jest dokładnie -- jak sugeruje nazwa algorytmu -- $k$ najbliższych sąsiadów.
        Współczynnik $k$ jest parametrem algorytmu, który musi zostać określony przed jego uruchomieniem.
        Ma ona duży wpływa skuteczność działania i poprawność klasyfikacji, dlatego też dobór odpowiedniej wartości $k$ jest bardzo ważny i często wykonywany wielokrotnie w celu znalezienia najlepszej wartości.

        Największym problemem \emph{k-najbliższych sąsiadów} jest jego bardzo szybko rosnąca złożoność wraz z rosnącą trudnością zadań do wykonania.
        Złożoność czasowa algorytmu dla pojedynczego punktu w przestrzeni wynosi $O(nd)$, gdzie $n$ to liczba wszystkich elementów zbioru danych wejściowych wykorzystywanych do uczenia systemu, a $d$ liczba wykorzystywanych cech reprezentowanych w przestrzeni problemowej, czyli inaczej liczba jej wymiarów \cite{laviale2023deep}.
        Wynika to z faktu, że dla każdego punktu znajdującego się w przestrzeni problemowej algorytm musi obliczyć odległość między nim a każdym innym punktem w zbiorze danych.
        Wraz ze wzrostem liczby cech lub rozmiaru zbioru danych, złożoność obliczeniowa \emph{k-nn} również znacznie wzrasta, co czyni ją kosztowną obliczeniowo i niepraktyczną w przypadku dużych zbiorów danych.
        Dodatkowo, znalezienie optymalnej wartości $k$ metodą siłową może również zwiększyć złożoność czasową algorytmu.

        Warto jednak również zaznaczyć, że sama idea stojąca za algorytmem \emph{k-nn} -- z natury dość prosta i intuicyjna -- bardzo dobrze sprawdza się w zastosowaniach nieco odbiegających od opisanego tutaj podejścia klasyfikacyjnego.
        Istnieje wariancja algorytmu oparta na uczeniu nienadzorowanym, \emph{nienadzorowana regresja K-najbliższych sąsiadów}.
        Zamiast próby nauczenia takiego systemu istniejących klas na bazie podanego zbioru danych wejściowych, algorytm ten próbuje samodzielnie znaleźć podobieństwa między danymi wejściowymi i grupować je w klastry, swojego rodzaju własne klasy, do których uważa, że dane powinny przynależeć.
        Sprawdza się to bardzo dobrze w problemach ekstrakcji cech, lub analogicznie także w zadaniach obniżenia wymiarowości przestrzeni problemowej \cite{wang2015accelerating}.
        W ten sposób nieskomplikowany algorytm \emph{k-najbliższych sąsiadów} służy jako pomoc dla algorytmów znacznie bardziej złożonych i zdolnych do rozwiązywania bardziej skomplikowanych problemów poprzez redukcję wymiarowości cech danych wejściowych.

\end{itemize}

\subsection{Uczenie głębokie i sieci neuronowe}

Bardzo duży rozgłos w ostatnich czasach zdobyły techniki uczenia maszynowego oparte na sieciach neuronowych (ang. \emph{neural networks}, \emph{NN}), w szczególności na sieciach neuronowych głębokich (ang. \emph{deep neural networks}, \emph{DNN}).
Są bardzo uniwersalne i potrafią rozwiązywać wiele problemów z bardzo szerokiego wachlarza obszarów i domen dziedzinowych.

\subsubsection{Podstawy sztucznych sieci neuronowych uczenia głębokiego}

Pierwotnie zaproponowany przez Warrena McCullocha and Waltera Pitts'a w 1943 roku \cite{mcculloch1943logical} model sztucznego neuronu był w rzeczywistości funkcją przyjmującą wiele wartości binarnych na wejściu i zwracającą jedną wartość binarną na wyjściu.
Przypominać to miało działanie prawdziwych biologicznych neuronów, gdzie każdy może być pobudzony (odpowiednik wartości $1$) lub nie (odpowiednik $0$).
Bazując na pobudzeniach swoich sąsiadów, z którymi jest połączony, neuron sam może zostać pobudzony i przekazać sygnał dalej.

Pomysł ten następnie rozwinął Frank Rosenblatt w 1957 roku, który zaproponował model \emph{perceptronu}, czyli sztucznego neuronu, który nieco odchodził od swojej biologicznej inspiracji i pozwalał na przekazywanie na wejściu wartości nie tylko binarnych, ale dowolnych rzeczywistych.
One z kolei były wymnażane przez odpowiadające im wagi połączeń (dodatnie dla pobudzających, ujemne dla hamujących)i sumowane, a następnie porównywane do ustalonego biasu i na tej podstawie neuron zwracał wartość binarną na wyjściu \cite{rosenbaltt1957perceptron}.

Pierwsze próby wykorzystania perceptronów do zadań klasyfikacyjnych miały mieszane rezultaty, ponieważ okazało się, że nie są one w stanie rozwiązać problemów, które nie są liniowo separowalne.
Sprawdzały się znakomicie w problemach regresyjnych, natomiast w zadaniach klasyfikacyjnych, w których dane nie mogły zostać podzielone na dwie klasy za pomocą prostej linii, nie były w stanie osiągnąć zadowalających wyników.

W rzeczywistości jednak problemem okazało się wykorzystanie tylko jednej \emph{warstwy} perceptronów, gdzie te same neurony otrzymywały dane wejściowe i były odpowiedzialne za zwrócenie poprawnych danych wyjściowych.
Znacznie lepiej zaczęły sobie radzić sieci, gdzie neuronów dodano więcej i zorganizowano je w odpowiedni sposób.
Wewnątrz takiej sztucznej sieci neuronowej neurony są uporządkowane w warstwy, czyli zbiór sztucznych neuronów ustawionych do siebie ``równolegle'', niepołączone wzajemnie między sobą, ale połączone z neuronami z warstw sąsiednich.
I tak wszystkie wyjścia neuronów warstwy niższej są połączone z jednym wejściem każdego z neuronów warstwy wyższej.
Warstwa wyjściowa nie posiada żadnych wyjść połączonych z następnymi neuronami, a jedynie zwraca dane wyjściowe całej sieci.
Czasem rozróżnia się także warstwę wejściową, która nie posiada żadnych wejść, a jedynie otrzymuje dane wejściowe.
W sieciach jednowarstwowych warstwa wejściowa jest jednocześnie warstwą wyjściową, ponieważ nie ma żadnych warstw pośrednich \cite{bishop1994neural}.

Nawet sieci dwuwarstwowe -- czyli takie z jedną warstwą pośrednia, nazywaną też często \emph{warstwą ukrytą} (ang. \emph{hidden layer}) -- są w stanie rozwiązać znacznie bardziej złożone problemy \cite{huang2000classification}.
Jak również można się spodziewać, dodanie kolejnych warstw ukrytych w dalszy sposób zwiększa możliwości sieci neuronowych.
W szczególności zwiększa ich zdolności do generalizowania problemu, do którego rozwiązywania są szkolone \cite{thomas2017two}.

Ten trend był kontynuowany gdy okazało się, że coraz głębsze sieci prowadzą do możliwości rozwiązywania coraz bardziej skomplikowanych problemów.
Wykorzystanie takich sieci nazywa się uczeniem głębokim (ang. \emph{deep learning}), a one same są nazywane głębokimi sieciami neuronowymi (ang. \emph{deep neural networks}, \emph{DNN}).

Okazuje się, że zarówno obliczanie wyjścia na bazie danych wejściowych (nazywane procesem \emph{feedforward}) może być zareprezentowane nie tylko przez śledzenia każdego neuronu z osobna, ale również przez operacje na wektorach i dwuwymiarowych macierzach (lub trójwymiarowych tensorach dla niektórych rodzajów użytych warstw neuronowych).
Aktywacje wszystkich neuronów jednej warstwy reprezentowane są prez wektor, a wagi połączeń między neuronami dwóch warstw -- przez macierz.
Wtedy iloczyn macierzowy wektora aktywacji warstwy wejściowej i macierzy wag połączeń między warstwą $n$ a warstwą $n+1$ daje po przepuszczeniu każdej wartości przez funkcję aktywacji wektor aktywacji neuronów warstwy $n+1$.

Przedstawiając wektor aktywacji warstwy $n$ jako $a^{(n)}$, macierz wag między nią a warstwą $n+1$ jako $W^{(n+1)}$, wektor biasów warstwy $n+1$ jako $b^{(n+1)}$, wektor sum ważonych na wejściach warstwy $n+1$ jako $z^{(n+1)}$ oraz funkcję aktywacji warstwy $n+1$ zastosowaną na każdym elemencie wektora jako $f$, można zapisać pełne obliczenie dla warstwy $n+1$ jako:

\[
  \mathbf{z}^{(n+1)} = \mathbf{W}^{(n+1)} \cdot \mathbf{a}^{(n)} + \mathbf{b}^{(n+1)}
\]
\[
  \mathbf{a}^{(n+1)} = f(\mathbf{z}^{(n+1)})
\]

Aby zastosować to do całej sieci neuronowej, należy zastosować to iteracyjnie zaczynając od danych wejściowych jako pseudo-aktywacji pewnych nieistniejących neuronów warstwy wejściowej.

Algorytm uczenia nadzorowanego takich sieci nazywa się \emph{propagacją wsteczną błędu} (ang. \emph{backpropagation of error}).
Po wykonaniu przywidywania bazującego na danych wejściowych, sieć neuronowa porównuje wynik z oczekiwanym wyjściem i oblicza błąd.
Następnie błąd ten jest propagowany wstecz, od warstwy wyjściowej do warstwy wejściowej, a wagi połączeń między neuronami są aktualizowane w taki sposób, by zmniejszyć błąd \cite{rojas1996backpropagation}.

Co ważne, możliwość reprezentowania aktywacji i wag zbiorowo w postaci wektorów i macierzy oraz obliczenia bazujące na ich dodawaniu i mnożeniu pozwalają na wykorzystanie technik numerycznych do optymalizacji obliczeń.
Wykonywanie takich obliczeń może być zrównoleglone, co znacznie przyspiesza wykonywanie obliczeń.

\subsubsection{Rodzaje warstw używanych w uczeniu głębokim}

Z poziomu paktycznego przy wykorzystywaniu sztucznych sieci neuronowych -- w szczególności przy uczeniu głębokim -- przeważnie operuje się na już zdefiniowanych rodzajach warstw, które mają już zaimplementowane odpowiednie zachowania, działania, funkcje aktywacji i optymalizacje do swojego zastosowania.

Niektóre z najczęściej wykorzystywanych rodzajów warstw głębokich sieci neuronowych to:

\begin{itemize}

  \item \emph{Warstwa gęsta} (ang. \emph{Dense layer}, czasem \emph{Fully Connected Layer})

        Jest to najbardziej podstawowy rodzaj warstwy w głębokich sieciach neuronowych, w dużej mierze opisany w poprzedniej sekcji.
        Główną cechą warstwy gęstej jest to, że każdy neuron w tej warstwie jest połączony z każdym neuronem w poprzedniej warstwie.
        Innymi słowy, wyjścia wszystkich neuronów poprzedniej warstwy są wejściami dla każdego neuronu w warstwie gęstej.

        Duże warstwy są bardzo dobre w generalizacji problemu dowolnego rodzaju, ale są kosztowne obliczeniowo i mogą prowadzić do przeuczenia (ang. \emph{overfitting}) oraz problemów
        zanikającego gradientu.
        Mimo to stanowią trzon większości głębokich sieci neuronowych, zazwyczaj wspierane przez inne, bardziej wyspecjalizowane warstwy.
        Pojawiają się najczęściej w strategicznych miejscach sieci, gdzie cechy wygenerowane / wyodrębnione przez inne warstwy powinny zostać w jak najlepszym stopniu wykorzystane, zamodelowane i przekazane dalej \cite{josephine2021impact}.

  \item \emph{Warstwa konwolucyjna} (ang. \emph{Convolutional Layer})

        Bardzo popularny rodzaj warstwy w glębokich sieciach neuronowych, szczególnie dobrze radzący sobie z ekstrahowaniem lokalnych cech danych przestrzennych, takich jak obrazy, co sprawia, że są niezastąpione w zadaniach związanych z analizą wizualną \cite{albawi2017understanding}.

        Główną cechą warstwy konwolucyjnej jest operacja konwolucji, która polega na przesuwaniu tak zwanego jądra konwolucyjnego (inaczej filtru) po danych wejściowych, takich jak obraz.
        Jądro to jest macierzą wag, która jest zdefiniowana przez projektanta modelu i jest uczona podczas procesu treningu.

        Operacja konwolucji umożliwia sieciom neuronowym wykrywanie krawędzi, tekstur, wzorców geometrycznych i innych cech charakterystycznych obiektów na obrazie.
        Ponadto, stosowanie wielu warstw konwolucyjnych o różnych rozmiarach filtrów pozwala na wykrywanie coraz bardziej złożonych i abstrakcyjnych cech.

  \item \emph{Warstwa rekurencyjna} (ang. \emph{Recurrent Layer})

        Rodzaj szerszej grupy warstw używanych w sieciach neuronowych do przetwarzania sekwencji danych, takich jak sekwencje tekstowe, dźwiękowe lub dane czasowe.
        W odróżnieniu od warstw konwolucyjnych, które są doskonale przystosowane do przetwarzania danych przestrzennych, warstwy rekurencyjne mają zdolność uwzględniania kontekstu historycznego i przetwarzania danych w kontekście sekwencyjnym.

        Warstwy rekurencyjne są zaprojektowane w taki sposób, aby umożliwiały przekazywanie informacji wstecz między krokami czasowymi lub elementami sekwencji.
        Dzięki temu sieć może ``pamiętać'' informacje z poprzednich kroków i uwzględniać je przy przetwarzaniu obecnej informacji.

        Ze względu na ich naturę są bardzo częstym wyborem przy zadaniach przetwarzania języka naturalnego (ang. \emph{natural language processing}, \emph{NLP}), takich jak rozpoznawanie mowy, tłumaczenie maszynowe, generowanie tekstu, czy też analiza sentymentu.
        Używane są także w zadaniach przetwarzania danych czasowych, takich jak prognozowanie cen akcji, czy też analiza danych medycznych.
        Jednak niespecjalnie nadają się do zastosowań w klasyfikacjach obrazów.

  \item \emph{Warstwa LSTM} (\emph{Long Short-Term Memory})

        Jest to konkretny implementacja z grupy warstw rekurencyjnych stosowana w sieciach neuronowych do analizy sekwencji danych.
        Została zaprojektowana, aby radzić sobie z problemem zanikającego gradientu pojawiającego się w warstwach rekurencyjnych i umożliwiać modelom przechowywanie i wykorzystywanie informacji długoterminowej w danych sekwencyjnych.

        Tak jak inne warstwy rekurencyjne, warstwa \emph{LSTM} jest skutecznym narzędziem w analizie sekwencji danych, takich jak język naturalny czy dane czasowe, ponieważ jest w stanie przechowywać informacje na dłuższe okresy i uwzględniać zależności na różnych odległościach czasowych.
        Dzięki mechanizmom bramkowym, \emph{LSTM} może nauczyć się wybierać, które informacje są ważne i które powinny zostać pominięte w analizie sekwencji, co czyni ją jedną z kluczowych innowacji w dziedzinie rekurencyjnych sieci neuronowych \cite{staudemeyer2019understanding}.

  \item \emph{Warstwa normalizacji partii} (ang. \emph{Batch Normalization Layer})

        To warstwa stosowana w sieciach neuronowych, która ma na celu poprawę stabilności procesu uczenia poprzez normalizację aktywacji w warstwach pośrednich.
        Ich wykorzystanie znacznie przyczyniło się do zwiększenia skuteczności i szybkości treningu głębokich sieci neuronowych.

        Podczas treningu głębokich sieci neuronowych, wartości aktywacji w warstwach pośrednich (na przykład warstwach ukrytych) mogą zmieniać się znacząco w miarę postępu w procesie uczenia.
        To może prowadzić do problemu zwanych "zmierającej aktywacji", co z kolei może spowolnić lub utrudnić proces uczenia się \cite{bjorck2018understanding}.
        Warstwa normalizacji partii wprowadza normalizację aktywacji dla każdej próbki w batchu treningowym.

        Może być ona stosowana w różnych typach sieci neuronowych razem z wieloma innymi warstwami, taki jak warstwy gęste lu konwolucyjne.
        Dzięki jej zastosowaniu, trening głębokich sieci staje się bardziej stabilny i wydajny, a modele często uzyskują lepsze wyniki na zbiorze walidacyjnym i testowym.

  \item \emph{Warstwa poolingowa} (ang. \emph{Pooling Layer})

        Rodzaj warstwy stosowanej głównie w konwolucyjnych sieciach neuronowych (CNN) w celu zmniejszenia wymiarów map cech generowanych przez wcześniejsze warstwy konwolucyjne.

        Najczęściej pojawia się bezpośrednio po warstwie konwolucyjnej, gdzie pomaga w redukcji liczby parametrów oraz obliczeń, co z kolei przyspiesza uczenie się i obliczenia w sieci.
        Takie zmniejszenie wymiarowości możne być wykonywane na przykład poprzez wybieranie największej wartości z określonego obszaru,jak robią to warstwy \emph{max-pooling}, lub poprzez uśrednianie wartości z określonego obszaru, jak robią to warstwy \emph{average-pooling}.

  \item \emph{Warstwa dropoutu}

        Jej użycie to technika stosowana w sieciach neuronowych, która pomaga w zapobieganiu przeuczeniu poprzez losowe wyłączanie pewnych neuronów w trakcie treningu.
        Jest to jeden z popularnych i skutecznych sposobów regularyzacji modelu.

        Główna idea warstwy dropoutu polega na tym, że w każdej iteracji treningowej, podczas propagacji wstecznej, losowo ``upuszczane'' (ang. ``dropped out'') są niektóre neurony w danej warstwie w celu redukcji zależności między nimi.
        Innymi słowy, neuron, który zostaje upuszczony, nie jest brany pod uwagę podczas propagacji wstecznej i aktualizacji wag \cite{srivastava2014dropout}.

        Warto zauważyć, że warstwa dropoutu jest stosowana tylko podczas treningu, a nie podczas predykcji.
        W jej trakcie wszystkie neurony są aktywowane, ale ich wagi są skalowane przez prawdopodobieństwo wyłączenia (ang. dropout rate), aby zrównoważyć wpływ warstwy dropoutu na model.

\end{itemize}

Z wymienionych powyżej rodzajów, w zadaniach klasyfikacji i detekcji w obrazach nie zazwyczaj są wykorzystywane jedynie warstwy rekurencyjne i LSTM, które mają swoje zastosowania raczej w problemach przetwarzania języka naturalnego i danych czasowych.
Wszystkie pozostałe są często wykorzystywane w budowaniu głębokiego modelu sieci neuronowej tak, aby zmaksymalizować skuteczność jego uczenia oraz późniejszą dokładność predykcji lub klasyfikacji.

\subsection{Wykorzystanie konwolucyjnych sieci neuronowych}

Do zadań analizy obrazów, w tym detekcji i ich klasyfikacji, najczęściej wykorzystywane są głębokie konwolucyjne sieci neuronowe.

\subsubsection{Podstawy konwolucyjnych sieci neuronowych}

\emph{Konwolucyjne sieci neuronowe} (ang. \emph{Convolutional Neural Networks}, \emph{CNNs}) są specjalnie zaprojektowanymi rodzajami sieci neuronowych, które doskonale sprawdzają się w analizie danych przestrzennych, takich jak obrazy czy dane wideo.

Ich głównym celem jest wykrywanie oraz ekstrahowanie cech o różnym stopniu złożoności w danych wejściowych.
Są to w rzeczywistości po prostu głębokie sieci neuronowe, które posiadają w swojej architekturze warstwy konwolucyjne.
Jednak ich obecność niesie ze sobą także kilka dodatkowych zmian, między innymi dużą potrzebę dodania warstw poolingowych oraz strategiczne umiejscowienie warstw gęstych tak, aby mogły wykorzystać cechy wygenerowane przez warstwy konwolucyjne.

Same warstwy konwolucyjne operują na zestawie filtrów (jąder) konwolucyjnych.
Są to małe macierze, na przykład wielkości 5 x 5 lub 3 x 3, które są przesuwane po danych wejściowych, takich jak obraz, w celu wykrycia pewnych cech.
Każdy z tych filtrów ma swoje wagi, które są trenowane podczas procesu uczenia i każdy będzie miał je inne, a co za tym idzie będzie izolował inne cechy obrazu źródłowego, takie jak krawędzie czy określone kształty.

Proces konwolucji zaczyna się od przesuwania filtra po danych wejściowych.
W każdym kroku przesunięcia, elementy macierzy filtra mnożone są przez odpowiadające im elementy obszaru danych, na którym jest stosowany filtr.
W wyniku operacji konwolucji otrzymujemy pojedynczą wartość dla każdego przesunięcia filtra na danych wejściowych.
Te wartości są następnie organizowane w tzw. mapę cech (ang. feature map), która reprezentuje wykryte cechy w określonym regionie danych wejściowych.
Jeśli dane wejściowe były dwuwymiarowe, pojawia się dodatkowy wymiar, który reprezentuje wykryte cechy w danych wejściowych.
Dla obrazów, które reprezentowane są jako macierz dwuwymiarowa, na wyjściu warstwy konwolucyjnej pojawia się więc tensor trójwymiarowy, gdzie trzeci długość trzeciego wymiaru odpowiada liczbie filtrów konwolucyjnych użytych w warstwie.
Jeśli jednak wejściem jest tensor trójwymiarowy, na przykład obraz kolorowy z wartości RGB tworzącymi trzeci wymiar, to proces konwolucji jest wykonywany osobno dla każdego z kanałów, a wyniki są sumowane, tworząc pojedynczy trójwymiarowy tensor wyjściowy.

Ważną cechą warstw konwolucyjnych jest to, że wagi filtrów są współdzielone -- te same wagi jednego filtra są używane do analizy różnych obszarów danych obrazu źródłowego.
Ta właściwość umożliwia sieciom konwolucyjnym efektywne wykrywanie tych samych wzorców czy cech w różnych częściach danych wejściowych.

\subsubsection{Przykładowa architektura konwolucyjnej sieci neuronowej}

\begin{figure}[ht]
  \includegraphics[width=\textwidth]{convolutional-neural-network-architecture}
  \caption[Przykładowa architektura konwolucyjnej sieci neuronowej]{Przykładowa architektura konwolucyjnej sieci neuronowej zaadoptowana z artykułu Sumita Saha na portalu \url{saturncloud.io} \cite{saha2018comprehensive}.}
  \label{fig:convolutional-neural-network-architecture}
\end{figure}

Na \hyperref[fig:convolutional-neural-network-architecture]{rysunku \ref*{fig:convolutional-neural-network-architecture}} przedstawiona jest przykładowa architektura konwolucyjnej sieci neuronowej zaadoptowana z artykułu na portalu \url{saturncloud.io} \cite{saha2018comprehensive}.
Sieć prezentuje system klasyfikacji obrazów ręcznie pisanych cyfr arabskich ze zbioru danych MNIST \cite{mnist}.

W niej dane wejściowe obrazu o wymiarach $28 \times 28 \times 1$ (czyli traktując macierz o wymiarach $28 \times 28$ jako tensor) przekazywane są najpierw do warstwy konwolucyjnej z $n1$ filtrami o rozmiarze $5 \times 5$.
Każdy filtr generuje swoją wynikową macierz o wymiarach $24 \times 24 \times 1$, a przez fakt, że użyto $n1$ filtrów, otrzymujemy tensor o wymiarach $24 \times 24 \times n1$.

Następna warstwa sieci to warstwa poolingowa, a dokładniej typu \emph{max-pooling}, która zmniejsza wymiary danych wejściowych.
Posiada ona \emph{okno poolingu} o wymiarach $2 \times 2$, które przesuwane jest po danych wejściowych z krokiem $2$ (tak, aby okna na siebie nie nachodziły).
Z każdego okna wybierana jest największa wartość, która jest następnie zapisywana w macierzy wyjściowej.
W ten sposób z danych wejściowych o wymiarach $24 \times 24 \times n1$ otrzymujemy macierz o wymiarach $12 \times 12 \times n1$.

Kolejna warstwa jest ponownie warstwą konwolucyjną o $n2$ filtrach o rozmiarze $5 \times 5$.
Każdy z $n1$ kanałów danych wejściowych (trzeci wymiar tensora) jest przetwarzany osobno przez każdy z filtrów i dla każdego generowany jest wyjściowy tensor o wymiarach $8 \times 8 \times n2$.
Następnie wszystkie $n1$ tensory są sumowane, tworząc pojedynczy tensor o wymiarach $8 \times 8 \times n2$ jako wyjście tej warstwy konwolucyjnej.

Kolejna warstwa to znów warstwa \emph{max-pooling} o wymiarach okna $2 \times 2$, która zmniejsza wymiary danych wejściowych z $8 \times 8 \times n2$ do $4 \times 4 \times n2$.

Później następuje spłaszczenie danych wejściowych do wektora o długości $4 \times 4 \times n2 = 16n2$ i przekazanie go do warstwy gęstej o $n3$ neuronach.
Połączenie między warstwą spłaszczoną a gęstą również jest w pełni połączone -- każdy z neuronów warstwy gęstej jest połączony z każdym z neuronów warstwy spłaszczonej.

Ostatnia warstwa gęsta ma $10$ neuronów, co odpowiada liczbie klas, na które ma być dokonywana klasyfikacja i jest jednocześnie warstwą wyjściową sieci.

\subsubsection{Przykłady zastosowań w analizie obrazów i diagnostyce medycznej}

Ne względu na swoją uniwersalność i skuteczność, głębokie uczenie maszynowe znalazło zastosowanie w wielu dziedzinach, w tym w analizie obrazów i diagnostyce medycznej.
Część algorytmów regresyjnych wykorzystywana jest do detekcji anomalii w sekwencjach danych różnego typu jak rytmu bicia serca, w tym także rekurencyjne sieci neuronowe i LSTM \cite{fernando2021deep}.
Jednak głębokie konwolucyjne sieci neuronowe są najczęściej wykorzystywane do analizy obrazów i diagnostyki medycznej lub detekcji nieprawidłowości bazującej na danych obrazowych.
Takie zastosowania również są często wykorzystywane i aktywnie rozwijane, a już na te chwilę osiągają bardzo dobre wyniki i pomagajają w diagnozowaniu wielu chorób \cite{liu2017survey}.

Zakres zastosowań w medycynie, do których wykorzystywane są głębokie konwolucyjne sieci neuronowe jest bardzo szeroki.
Jest to przede wszystkim diagnostyka bazująca na obrazach rezonansu magnetyczne, tomografii komputerowej, czy też obrazach rentgenowskich.

Jednym z takich konkretnych zastosowań jest wykrywanie raka piersi na podstawie obrazów mammograficznych.
Badacze wykorzystali piramidową reprezentację Gaussa nazywaną też piramidową konwolucyjną siecią neuronową (ang. \emph{Pyramid-CNN}), wariant sieci CNN zbudowanej z coraz to mniejszych warstw konwolucyjnych przypominających strukturalnie piramidę \cite{bakkouri2019multi}.
Najszerszy poziom to warstwa wejściowa, a każda kolejna warstwa jest coraz węższa, aż do kategoryzujące warstwy wyjściowej.
Osiągnięte wyniki były bardzo obiecujące, gdzie system osiągnął średnią dokładność 96,84\%, czułość 92,12\%, swoistość 98,02\%, precyzję 92,15\%.

Innym przykładem zastosowań również z obszaru diagnostyki onkologicznej jest wykorzystnaie głębokich konwolucyjnych sieci neuronowych do detekcji raka trzustki na podstawie obrazów tomografii komputerowej \cite{sekaran2020deep}.
Przestawione tam również było porównanie różnych metod uczenia maszynowego do tego samego zadania i na tych samych danych, gdzie najlepiej ze wszystkich uwzględnionych algorytmów wypadło głębokie konwolucyjne uczenie maszynowe.

Kolejnym przykładem jest wykorzystanie głębokich konwolucyjnych sieci neuronowych w zbudowaniu endoskopowej bazy wiedzy, która może być używana w podejściu opartym na zapytaniach \cite{petscharnig2018binary}.
Godne zauważenia tutaj jest, że system był uczony na materiałach wideo i na takich również miał za zadanie operować.
Wykorzystując bazę nagrań endoskopowych gromadzonych w trakcie operacji przez chirurgów badaczom udało się zbudować system, który potrafił poprawnie odpowiadać na dobrze skonstruowane zapytania.
Pokazali również, że cechy zbudowanego przez nich modelu CNN są kompaktowymi, ale wydajnymi deskryptorami obrazu do wyszukiwania w domenie obrazowania endoskopowego.
Są one w stanie utrzymać najnowocześniejszą wydajność, zapewniając jednocześnie korzyści w postaci niskiego zapotrzebowania na miejsce do przechowywania, a tym samym zapewniają najlepszy kompromis.

Pomimo wielu pozytywów warto jednak również zauważyć, że medycyna jest dziedziną z niesamowicie małym marginesem błędu oraz gdzie nawet skuteczne rozwiązania nie będące w pełni przetestowane pod względem ich bezpieczeństwa i skuteczności są odrzucane.
Między innymi to jest powodem, dla którego wdrożenie rozwiązań głębokiego uczenia maszynowego jest tak czasochłonne i często budzi również kontrowersje.
Metody uczenia maszynowego nazywane są ``czarnymi skrzynkami'', ponieważ poza wyjaśnieniem doboru architektury, procesu uczenia sieci i algorytmu za nim stającym nie jest możliwe wyjaśnienie, dlaczego sieć dokonuje takich, a nie innych predykcji, co jest problematyczne w obszarze opierającym się na pełnej pewności bezpieczeństwa używanych rozwiązań \cite{pouyanfar2018survey}.

\subsection{Uczenie transferowe i wstępnie wytrenowane modele sieci neuronowych}

Uczenie transferowe to technika uczenia maszynowego, która polega na wykorzystaniu wiedzy zdobytej z jednego zadania w celu poprawienia wyników innego, pokrewnego zadania.
To potężne podejście stosowane w sytuacjach, gdzie dane dla celowego zadania są ograniczone, a dostępne są obfite dane dla zadania źródłowego, które jest powiązane.

Technika ta ma wiele zalet, przede wszystkim prędkość trenowania -- ``dotrenowanie'' modelu wykorzystujące inny, wstępnie wytrenowany model jest znacznie szybsze niż trening podobnego modelu od zera.
Wykorzystuje on bowiem wiedzę zgromadzoną w modelu źródłowym, który został już wcześniej wytrenowany na ogromnych zbiorach danych.
Dodatkowo efektywne wykorzystane są dane, trenowanie modelu używając uczenia transferowego wymaga znacznie mniej danych wejściowych niż trenowanie modelu od zera \cite{weiss2016survey}.

Zalety te opierają się na jednej zasadniczej cesze wstępnie wytrenowanego modelu -- jest on uczony w taki sposób, aby wytworzyć reprezentacje bardzo uniwersalne, które są w stanie uchwycić wiele różnych cech w danych wejściowych określonego typu.
Innymi słowy posiada już zgeneralizowaną wiedzę na temat ogólnej kategorii danych, dla których jest przeznaczony.
Przykładowo model dedykowany jako baza w kategoryzacji obrazów będzie miał umiał w bardzo ogólny sposób analizować obrazy prowadząc do takiego reprezentacji, aby posiadała jak najwięcej informacji o cechach obrazu w sposób uniwersalny.
Następnie doszkalany model bazowany na tym już wstępnie wytrenowanym wyłącznie wykorzystuje te bogate reprezentacje cech oferowane przez rdzenny model do kategoryzacji obrazów wybranego rodzaju, na przykład odróżnianie obrazów psów od kotów.

Modele wstępnie wytrenowane są zazwyczaj ogromnie i mają bardzo złożoną architekturę.
Zadanie generalizowania obszaru problemowego dla problemów, które nie są jeszcze znane jest bowiem bardzo trudne, co odzwierciedla wielkość takich sieci.

Dla przykładu jedną z popularnych wykorzystywanych wstępnie wytrenowanych sieci neuronowych stosowanych w zadaniach analizy obrazu jest \emph{ResNet} (\emph{Residual Neural Network}).
Jego architektura zaczyna się od standardowego bloku konwolucyjnego i normalizujacego, jednak następnie posiada kluczowe dla swojego działania specjalne \emph{bloki resztkowe} (inaczej \emph{rezydualne}, z ang. \emph{residual blocks}) \cite{li2016demystifying}.
Blok resztkowy składa się z co najmniej dwóch konwolucyjnych warstw, które przetwarzają dane wejściowe, a następnie ich wynik jest dodawany do oryginalnych danych wejściowych.
To dodanie pozwala na przekazywanie reszty (różnicy) między oryginalnymi danymi a przetworzonymi danymi przez blok resztkowy.
Dalej składa się ponownie z klasycznie używanych bloków takich jak warstwy poolingowe czy gęste, aż do warstwy wyjściowej.

Dzięki swojej blokowej budowie w zależności od potrzeb można wykorzystać różne warianty sieci ResNet, które różnią się głębokością sieci.
Dostępne są modele \emph{ResNet-18}, \emph{ResNet-34}, \emph{ResNet-50}, \emph{ResNet-101}, \emph{ResNet-152} oraz \emph{ResNet-200}, gdzie liczba w nazwie oznacza liczbę warstw w sieci.

\chapter{Metodologia i narzędzia uczenia maszynowego środowiska .NET}

Celem tej pracy jest wykorzystanie technologii uczenia maszynowego w środowisku .NET w celu detekcji choroby Alzheimera.
W tym rozdziale więc zostaną przedstawione narzędzia, które pomogą w osiągnięciu tego celu, w tym technologie uczenia maszynowego oraz sama platforma .NET.

\section{Technologie głębokiego uczenia maszynowego}

Teoretyczne zagadnienia związanie z technologiami głębokiego uczenia maszynowego, w szczególności głębokich konwolucyjnych sieci neuronowych zostały już omówione w \hyperref[sec:deep-learning]{sekcji \ref*{sec:deep-learning}}.
W tym rozdziale zostanie przedstawione w jaki sposób obiecująca teoria jest przekładana na wykorzystanie w praktyce.

\subsection{Symulator SNNS}

Tematykę oprogramowania służącego do tworzenia i trenowania sieci neuronowych warto zacząć omówienia przestarzałego już programu \emph{SNNS} (ang. Stuttgart Neural Network Simulator).
Pozwala on na tworzenie i trenowanie sieci neuronowych, a także -- co ważne przy jego zastosowaniach -- na ich wizualizację.

SNNS to program okienkowy napisany w języku C, przeznaczony głównie dla systemów Unixowych.
W ramach jego działania można zamodelować sieć neuronową po jednym neuronie, łącząc je z dużą dozą swobody w dowolne struktury oraz parametryzując je w dowolny sposób.
Następnie można zdefiniować zbiór danych uczących, a także zbiór danych testowych, na których można przeprowadzić proces uczenia sieci.

Całe tworzenie sieci jest czynnością bardzo czasochłonną i złożoną ze względu na manualną naturę definiowania jej struktury.
Daje za to bardzo dużo możliwości modyfikacji, w tym zmianę na przykład funkcji aktywacji pojedynczego neuronu, zachowania się konkretnego połączenia oraz wielu innych właściwości, które posiadają wszystkie obiekty możliwe do edycji w programie.

Najważniejszą cechą SNNS jest wizualny sposób budowy sieci neuronowej oraz manualny proces jej konfiguracji.
Nie jest ona może wtedy wystarczająco wydajna do jakichkolwiek problemów, z którymi potrafią radzić sobie nowoczesne sieci neuronowe, ale pozwala na zrozumienie ich działania oraz nauczenie się podstawowych zasad ich budowy.
Dlatego też mimo, iż sam program jest już przestarzały i wycofany z użytku a sami jego autorzy polecają użycie nowoczesnych bibliotek uczenia maszynowego takich jak TensorFlow czy PyTorch \cite{snns}, to nadal jest bardzo często wykorzystywany w celach edukacyjnych.

Pomimo, iż znane są metody znacznie przyspieszające działanie i uczenie sieci neuronowych przez reprezentacje wag i aktywacji jako macierzy i wektorów i ich późniejsze mnożenie przy pomocy zrównoleglonych obliczeń na kartach graficznych i dedykowanych urządzeniach, to założenia podstaw działania sztucznych sieci neuronowych pozostają niezmienne i programy typu SNNS pozwalają na znacznie prostsze ich poznanie i zrozumienie.

\subsection{Azure Machine Learning Studio}

\emph{Microsoft Azure Machine Learning Studio} (w skrócie \emph{AML Studio}) to kompleksowa platforma, która wykorzystywana jest do implementacji i zarządzania procesami uczenia maszynowego.
AML Studio oferuje zaawansowane narzędzia do tworzenia, wdrażania oraz monitorowania modeli uczenia maszynowego, integrując różnorodne etapy tego procesu w jednym środowisku.

Jest to platforma oparta na chmurze i posiadająca interfejs graficzny w postaci intuicyjnej strony internetowej, na której modelowanie przetwarzania danych i procesu uczenia odbywa się przy użyciu przeciągania i upuszczania elementów blokowych oraz łączenia ich w odpowiedni sposób.
Pozwala na tworzenie tak zwanych \emph{eksperymentów}, które składają się z kolejnych kroków w analizie danych i tworzeniu modeli.
Dzięki temu możliwe jest systematyczne badanie różnych podejść i łatwe porównywanie ich wyników.
Graficzny sposób reprezentacji wykorzystanych modułów w eksperymencie a także przepływu danych między nimi pozwala na uproszczenie wyszukiwania potencjalnych błędów lub problemów pojawiających się w całym procesie.
Możliwe jest również analizowanie w czasie rzeczywistym działania modelu i analizę przez niego danych w całym eksperymencie.

Bardzo istotną cechą AML Studio jest fakt, że platforma integruje się z innymi usługami chmurowymi dostarczanymi przez Microsoft Azure, co umożliwia tworzenie spójnych rozwiązań opartych na chmurze.
Pozwala na dynamiczne wczytywanie danych z innych źródeł znajdujących się w chmurze, a także na wdrażanie modeli uczenia maszynowego w postaci usług sieci Web, które mogą być wykorzystywane przez inne aplikacje.

Wykorzystanie Azure Machine Learning Studio dzięki swoim cechom pozwala na szybki trening modelu uczenia maszynowego oraz jego wdrożenie bez konieczności pisania kodu czy głębszej znajomości tematyki uczenia maszynowego \cite{mukunthu2019practical}, w szczególności, gdy z założenia wdrożony model ma działać w chmurze i integrować się z innymi systemami w niej obecnymi.
Jednak w zastosowaniach, które nie czerpią korzyści z połączenia z chmurą i dają swobodę czasową na własnoręczne zaimplementowanie modelu, wykorzystanie AML Studio może być nieopłacalne ze względu na koszty związane z jego wykorzystaniem w większych ilościach.
Warto wtedy rozważyć inne rozwiązania, które dają swobodę konstruowania i trenowania modelu od podstawi i dostrojenie go do bardziej złożonych zadań, maksymalizując tym samym osiągane możliwości i dokładność -- a są to zazwyczaj biblioteki uczenia maszynowego w językach programowania.

\subsection{TensorFlow i Keras}
\label{sec:tensorflow-and-keras}

Jedną z najpopularniejszych bibliotek uczenia maszynowego, która jest wykorzystywana do tworzenia i trenowania sieci neuronowych jest \emph{TensorFlow}.
Jest to otwartoźródłowa platforma do obliczeń numerycznych i implementacji modeli uczenia maszynowego.
Jego fundamentem jest reprezentacja danych w postaci tensorów, co umożliwia wykonywanie skomplikowanych operacji matematycznych na dużych zbiorach danych \cite{shukla2018machine}.

TensorFlow jest biblioteką niskopoziomową napisaną głównie w języku C++ i implementującą często używane algorytmy uczenia maszynowego i sieci neuronowych i wystawiającą je jako interfejs programistyczny w językach wyższego poziomu takich jak Python czy JavaScript.
Napisany został pierwotnie przez zespół badawczy Google Brain.

Jednak ze względu swoje dążenie do umożliwienia jak największej elastyczności i kontroli nad procesem uczenia sieci neuronowych, TensorFlow w swojej czystej postaci wymaga od programisty dużo pracy i wiedzy, aby zaimplementować nawet najprostszy model sieci neuronowej.

W celu uproszczenia z korzystania z TensorFlow powstały ``frontendy'' dla różnych języków programowania pozwalające na wykorzystanie jego możliwości w bardziej przyjazny sposób.
Jednym z nich jest \emph{Keras}, który jest wysokopoziomowym interfejsem programistycznym dla TensorFlow w języku Python.
Jest to jeden z najpopularniejszych interfejsów programistycznych dla TensorFlow ze względu na swoją prostotę i intuicyjność.

Dla uproszczenia wystawia API sekwencyjne (ang. Sequential API), które przez wykorzystanie istniejących klas i ``sekwencyjne'' tworzenie obiektów symbolizujących określone rodzaje warstw ze zdefiniowanymi odpowiednio parametrami pozwala na proste w śledzeniu i zrozumieniu stworzenie struktury sieci neuronowych.
Operacje takie jak \emph{skompilowanie} (czyli utworzenie z definicji warstw sieci zoptymalizowanego do uruchomienia i uczenia wstępnego modelu) definicji modelu używając odpowiednich parametrów czy uruchomienie trenowania (ang. \emph{fit}) w odpowiedniej konfiguracji wykonuje się wywołaniem tylko  jednej metody.

Mimo swojej prostoty, Keras pozwala na również na podejścia bardziej eksperckie.
W nich oprócz definiowania sekwencyjnego struktury sieci można również tworzyć własne klasy reprezentujące niestandardowe warstwy lub całe bloki odpowiednio ustrukturyzowane i sparametryzowane.

Uczenie sieci neuronowej w Keras zachowuje również zbiór metryk, które pozwalają na prześledzenie postępów w procesie uczenia i wykorzystanie ich do analizy lub optymalizacji modelu albo całego schematu programu.

Sukces TensorFlow oraz Keras wynika głównie z połączenia ich wydajności, elastyczności i prostoty użycia, które w dużej mierze są zasługą również wykorzystanego języka programowania Python, najpopularniejszego do zastosowań w uczeniu maszynowym.
Jednak ze względu na to, że obie te biblioteki mają swoje źródła dostępne na bazie licencji Apache 2.0 w publicznie dostępnym repozytorium serwisu GitHub \cite{tensorflow}.
Pozwala to społeczności na własne próby rozszerzenia zasięgu platformy do innych języków programowania, na przykład jak zostanie opisane w \hyperref[sec:tensorflownet]{rozdziale \ref*{sec:tensorflownet}} -- do języka C\#.

\section{Platforma .NET}

Platforma .NET jest kompleksowym środowiskiem programistycznym opracowanym przez pierwotnie firmę Microsoft, a obecnie rozwijanym przez niezależną organizację non-profit \emph{.NET Foundation} oraz społeczność programistów.

W celu lepszego zrozumienia aktualnego stanu platformy warto przyjrzeć się jej historii i poprzednikowi, którym jest \emph{.NET Framework}.
Jest to starsza wersja, która była szeroko wykorzystywana przed pojawieniem się \emph{.NET Core} (i nadal jest w wielu przestarzałych bazach kodu określanych jako \emph{legacy code}).
Działała ona wyłącznie na systemach operacyjnych Windows i była bardzo ściśle z nimi związana.
Miał bardzo szerokie zastosowania, pozwalał na pisanie aplikacji okienkowych, konsolowych, usług systemowych czy aplikacji sieciowych z użyciem różnych języków programowania -- głównie najpopularniejszego obiektowego języka C\#, ale także Visual Basic czy funkcjonalnego F\#.
\emph{.NET Framework} został wypuszczony w 2002 jako konkurent dla platformy Java i jej maszyny wirtualnej JVM, która również pozwala na pisanie aplikacji w wielu językach programowania i jest niezależna od systemu operacyjnego.
Podobnie jak ona kod kompilowany jest do kodu pośredniego, który jest wykonywany przez maszynę wirtualną \emph{Common Language Runtime} (CLR) -- odpowiednika JVM w Javie.
Kod ten nazywa się \emph{IL} (ang. \emph{Intermediate Language}, odpowiednik \emph{Java bytecode}) i jest zapisywany w plikach o rozszerzeniu \emph{.dll} (ang. \emph{Dynamic Link Library}) lub \emph{.exe} (ang. \emph{Executable})
Tak skompilowany kod jest później uruchamiany przez interpreter CLR, podobnie jak ma to miejsce w przypadku JVM (\emph{Java Virtual Machine}).

W 2014 roku Microsoft ogłosił, że rozpoczyna prace nad nową wersją platformy .NET, którego rozwój ma odbywać się pod nadzorem nowej, niezależnej organizacji non-profit \emph{.NET Foundation}.
Wersja stabilna 1.0 nowego tworu nazwanego \emph{.NET Core} została wydana w 2016 roku jako platforma w pełni otwartoźródłowa, międzyplatformowa i modułowa.
Oznacza to, że jest ona dostępna na systemach operacyjnych Windows, Linux i macOS, a także że jej komponenty są dostępne jako osobne pakiety, które można wykorzystywać w zależności od potrzeb.
Cały kod źródłowy jest dostępny już na portalu GitHub, upubliczniony z bardzo liberalną licencją MIT \cite{dotnet-sdk-repo, dotnet-runtime-repo}, wraz ze wszystkimi bibliotekami wbudowanymi służącymi do tworzenia różnych rodzajów oprogramowania, a nawet repozytorium służące do rozwijania składni języka wraz z wkładem od społeczności.
Od tamtego czasu nowa wersja platformy poza jednym wyjątkiem była wypuszczana co roku.
W 2020 roku, po poprzednim wydaniu .NET Core 3.1 ogłoszono, że pominięta zostanie wersja 4 ze względu na możliwe mylenie jej z działającą wtedy jeszcze w wielu miejscach wersją 4.8 starszego \emph{.NET Framework}.
Dodatkowo celu unifikacji nazewnictwa oraz sprostowanie, że .NET Core nie jest okrojoną wersją, ale pełnoprawną platformą, która jako jedyna będzie dalej rozwijana, zdecydowano się na pomięcie członu \emph{Core} i od wersji 5.0 platforma nazywa się po prostu \emph{.NET 5}.

\subsection{Język C\#}
\label{sec:csharp}

Język programowania C\# jest jednym z najpopularniejszych języków programowania na platformie .NET.

Jest to język obiektowy, który został zaprojektowany przez Microsoft w 2000 roku jako główny język platformy .NET.
Od tego czasu jest stale rozwijany i wraz z nowymi wersjami platformy .NET pojawiają się nowe wersje języka C\#, z najnowszą na moment pisania niniejszej pracy wersją \emph{C\# 11} wypuszczoną razem z \emph{.NET 7} w 2022 roku.
Rozwój języka przebiega wraz ze wkładem społeczności, która zgłasza propozycje nowych funkcjonalności i poprawek do istniejących przy użyciu oficjalnego repozytorium języka na portalu GitHub \cite{dotnet-csharplang-repo}, na którym dodatkowo znajduje się historia wszystkich proponowanych zmian oraz informacje na ich temat, a także notatki z wewnętrznych spotkań \emph{.NET Foundation} dotyczących dalszego projektu języka.

Kilka cech języka:

\begin{itemize}

  \item Bezpieczeństwo typów: C\# jest językiem silnie typowanym, co oznacza, że typy zmiennych są sprawdzane podczas kompilacji, co pomaga uniknąć wielu błędów w czasie działania programu.

  \item Automatyczne zarządzanie pamięcią: Automatyczne zarządzanie pamięcią: Dzięki mechanizmom takim jak Garbage Collection (zbieranie nieużywanej pamięci) C\# pozwala programistom uniknąć wielu problemów związanych z wyciekami pamięci.

  \item Asynchroniczność: C\# oferuje obszerną obsługę operacji asynchronicznych, co jest szczególnie ważne w aplikacjach wielowątkowych i sieciowych.

\end{itemize}

Dodatkowo język ten jest bardzo elastyczny i pozwala na wykorzystanie wielu paradygmatów programowania, w tym obiektowego, imperatywnego, funkcyjnego czy generycznego.
W szczególności ostatni rozwój języka brnie w kierunku paradygmatu funkcyjnego inspirowany językami takimi jak F\# -- również z rodziny \emph{.NET} -- czy Haskell.
Jednym z przykładów jest rosnąca popularność użycia ``fluent interfejsów'', czyli pisania metod w taki sposób, by pozwalały na ich łańcuchowe wywoływanie, co pozwala na pisanie kodu w sposób bardzo czytelny.

Przykładowo poniższy fragment kodu z projektu \lstinline{MLModel_ConsoleApp} napisanego w dodatku do tej pracy pozwala na zbudowanie potoku przetwarzania danych, który składa się z trzech kroków poprzez proste wywoływanie metod po sobie na własnych zwracanych obiektach.

\begin{lstlisting}[language={[Sharp]C}]
public static IEstimator<ITransformer> BuildPipeline(MLContext mlContext)
{
    // Data process configuration with pipeline data transformations
    var pipeline = mlContext.Transforms.Conversion.MapValueToKey(/*...*/)
        .Append(mlContext.MulticlassClassification.Trainers.ImageClassification(/*...*/))
        .Append(mlContext.Transforms.Conversion.MapKeyToValue(/*...*/));

    return pipeline;
}
\end{lstlisting}

W ten sposób kod jest czytelny i łatwy do zrozumienia, a dodatkowo pozwala na łatwe dodawanie kolejnych kroków do potoku bez konieczności dużej zmiany istniejącego kodu.
Jest to szczególnie przydatne do implementacji wzorców projektowych takich jak \emph{Builder} czy \emph{Decorator}, które w C\# są wykorzystywane bardzo często.

\subsection{Język F\#}

F\# jest językiem funkcyjnym, co oznacza, że skupia się na programowaniu funkcyjnym, w którym funkcje są traktowane jako podstawowe jednostki programu
Funkcje w F\# są traktowane jak wartości, co umożliwia bardziej zdeklarowane i modularne podejście do programowania.

Najważniejsze cechy języka:

\begin{itemize}

  \item Kompaktowość: Funkcyjny styl programowania często pozwala na bardziej zwięzły kod w porównaniu do tradycyjnych języków.

  \item Bezpieczeństwo typów: F\# jest językiem silnie typowanym, co pomaga w unikaniu błędów podczas kompilacji i działania programu.

  \item Współbieżność: Dzięki naturze funkcyjnej, F\# ma wbudowane mechanizmy ułatwiające programowanie współbieżne i równoległe.

  \item Pattern matching: Jest to potężna technika w F\#, która pozwala na efektywne przetwarzanie i rozpoznawanie różnych wzorców w danych.

\end{itemize}

F\# przynależy on do rodziny języków \emph{.NET}, przez co jest w pełni kompatybilny z pozostałymi językami i kompilowany jest do tego samego kodu \emph{IL} i można go wykorzystywać w taki sam sposób jak inne języki z tej rodziny, a paczki napisane w F\# mogą być wykorzystywane przez inne języki i na odwrót.
Dla przykładu w projekcie \lstinline{MLNetCustom} napisanym do tej pracy napisałem bibliotekę \lstinline{Plot} w języku F\#, ponieważ sekwencyjne tworzenie wykresów bardzo pasowało do funkcyjnego paradygmatu programowania.
Następnie bibliotekę tę wykorzystałem w głównym projekcie napisanym w języku C\# bez żadnych problemów z kompatybilnością.

\section{Framework uczenia maszynowego ML.NET}
\label{sec:ml-net}

\emph{ML.NET} to otwartoźródłowy i międzyplatformowy framework uczenia maszynowego dla platformy .NET.
Repozytorium z jego kodem źródłowym jest publiczne i udostępnione na portalu GitHub z licencją MIT \cite{dotnet-machinelearning-repo}.
Umożliwia programistom tworzenie modeli uczenia maszynowego w języku C\# lub F\# bez konieczności głębokiego zrozumienia matematyki lub statystyki.
Jest to narzędzie skierowane do programistów platformy .NET chcących wykorzystać uczenie maszynowe w swoich projektach bez potrzeby nauki innych języków programowania lub korzystania z zewnętrznych narzędzi.

Ważnym aspektem, który warto podkreślić, jest integracja ML.NET z istniejącą infrastrukturą .NET.
Pozwala to wykorzystać dostępne w .NET narzędzia i biblioteki do wczytywania danych, przetwarzania, wizualizacji i zarządzania modelem.
To sprawia, że praca z danymi i modelami jest bardziej spójna i efektywna.

ML.NET dostarcza zestaw gotowych algorytmów uczenia maszynowego, takich jak klasyfikacja, regresja, klasteryzacja, przetwarzanie języka naturalnego i wiele innych.
Można wybrać odpowiedni algorytm do swojego zadania bez konieczności implementowania go od zera.
Ważne spostrzeżenie tutaj to fakt, że ten zestaw daje odgórny podział na kategoria zadań do wykonania.
Nie pozwala na przykład zbudować warstwa po warstwie modelu głębokiej sieci neuronowej.
Zamiast tego wystawia klasy takie jak \lstinline{MulticlassClassification} dającą dostęp do zbioru już zaimplementowanych narzędzi i algorytmów klasyfikacji wieloklasowej, lub \lstinline{ImageClassificationTrainer} pozwalający na trenowanie modelu klasyfikacji obrazów.
Konfiguracja tego drugiego opiera się wyłącznie na zdefiniowaniu informacji dotyczących danych i hiperparametrów uczenia, takich jak wielkość paczki danych (ang. batch size), liczba epok (ang. epochs), czy współczynnik uczenia (ang. learning rate).
Dodatkowo można ustawić w jaki sposób współczynnik uczenia ma się zmieniać z czasem (ang. learning rate scheduler lub learning rate decay), czy testowanie powinno odbywać się także na zbiorze uczącym, kryteria zatrzymania uczenia (ang. early stopping criteria) czy metodę zwrotną (ang. callback method), która zostanie wywołana po każdej epoce uczenia.

Jedna z istotniejszych opcji konfiguracji obiektu \lstinline{ImageClassificationTrainer} to wybór architektury, na której bazie będzie przeprowadzane uczenie.
Jest tak dlatego, ponieważ klasyfikacja obrazów ML.NET działa na podstawie uczenia transferowego omówionego od strony teoretycznej w \hyperref[sec:transfer-learning]{sekcji \ref*{sec:transfer-learning}}.
Dostępne do wybrania architektury to \emph{InceptionV3}, \emph{MobilenetV2}, \emph{ResnetV250} oraz domyślna \emph{ResnetV2101}.
Takie podejście pozwala na szybkie zbudowanie modelu uczenia maszynowego bez konieczności głębokiego zrozumienia jego działania, skupiając się wyłącznie na oczekiwanym celu.

W ramach mojego badania jeden z projektów jest oparty o czy ML.NET budowany kodowo -- solucja \lstinline{MLNetCustom}.
Jednak ML.NET można wykorzystać także bez bezpośredniego pisania kodu dotyczącego modelu i uczenia z użyciem rozszerzenia do Visual Studio \emph{ML.NET Model Builder}.

\subsection{ML.NET Model Builder}
\label{sec:ml-net-model-builder}

\emph{ML.NET Model Builder} to darmowe rozszerzenie do środowiska programistycznego Visual Studio, które pozwala na tworzenie modeli uczenia maszynowego bez konieczności pisania kodu.
Jest całkowicie oparte na wizualnym, okienkowym wybraniu odpowiednich opcji, danych czy scenariusza.

Używając instalatora Visual Studio użytkownik może zmodyfikować instalację wybierając dodatkowo komponent o nazwie \emph{ML.NET Model Builder}.
Pozwoli on da dodanie do projektu nowego elementu \emph{Machine Learning Model (ML.NET)}.
Element taki jest plikiem o rozszerzeniu \emph{.mbconfig} i pozwala na zdefiniowanie konfiguracji uczenia maszynowego do przeprowadzenia.

Następnie tak stworzony plik otworzony w karcie programu Visual Studio wyświetla wizualną reprezentację konfiguracji uczenia maszynowego i pozwala na wybieranie jej opcji krok po kroku.

\begin{figure}[ht]
  \includegraphics[width=\textwidth]{ml-net-model-builder-1-scenario}
  \caption{Zrzut ekranu ze środowiska Visual Studio pokazujący wybór scenariusza uczenia maszynowe w \emph{ML.NET Model Builder}}
  \label{fig:ml-net-model-builder-1-scenario}
\end{figure}

Takim pierwszym krokiem jest wybranie scenariusza uczenia maszynowego, które chcemy przeprowadzić w dla którego modelu chcemy wytrenować.
Wygląd takiego okna przedstawia \hyperref[fig:ml-net-model-builder-1-scenario]{rysunek \ref*{fig:ml-net-model-builder-1-scenario}}.
Do wyboru jest 7 scenariuszy z 3 większych kategorii dedykowanych pod różne zadania:

\begin{itemize}

  \item Kategoria \emph{Tabular data} -- dane tabelaryczne, czyli dane numeryczne lub tekstowe przedstawione w postaci tabeli, najczęściej w formacie CSV.
        Scenariusze w tej kategorii:

        \begin{itemize}

          \item \emph{Data classification} -- Klasyfikacja danych, służy do podziału danych na kategorie.
                Może to być klasyfikacja binarna lub wieloklasowa.
                Przykładowo klasyfikacja danych z popularnego zbioru o irysach, który na podstawie długości i szerokości płatka i działki kwiatu klasyfikuje obiekt jako jeden z trzech gatunków irysa.

          \item \emph{Value prediction} -- Przewidywanie wartości, wchodzi w zakres zadania regresji.
                Służy do przewidywania przyszłych wartości liczbowych na podstawie danych historycznych.
                Przykładowo przewidywanie ceny mieszkania na podstawie jego powierzchni i lokalizacji.

          \item \emph{Recommendation} -- Rekomendacja, przewiduje listę sugerowanych elementów dla konkretnej jednostki, w oparciu o to, jak podobne są jej upodobania do upodobań innych jednostek.
          Scenariusza rekomendacji można użyć, gdy masz zestaw  (użytkowników) i zestaw ``produktów'', takich jak przedmioty do zakupu, filmy, książki lub programy telewizyjne, wraz z zestawem ``ocen'' użytkowników tych produktów.

          \item \emph{Forecasting} -- Prognozowanie, wykorzystuje dane historyczne z szeregiem czasowym lub składnikiem sezonowym w celu na przykład przewidywania popytu lub sprzedaży produktu.

        \end{itemize}

  \item Kategoria \emph{Computer Vision} -- widzenie komputerowe, czyli przetwarzanie obrazów i wideo.
        Scenariusze w tej kategorii:
        \begin{itemize}

          \item \emph{Image classification} -- Klasyfikacja obrazów, służy do identyfikacji obrazów różnych kategorii.
                Na przykład różnych typów terenu, zwierząt lub wad produkcyjnych.
                Scenariusza klasyfikacji obrazów można użyć, jeśli masz zestaw obrazów i chcesz zaklasyfikować obrazy do różnych kategorii.

          \item \emph{Object detection} -- Wykrywanie obiektów służy do lokalizowania i kategoryzowania podmiotów na obrazach.
          Na przykład lokalizowanie i identyfikowanie samochodów i osób na obrazie.
          Wykrywania obiektów można używać, gdy obrazy zawierają wiele obiektów różnych typów.

        \end{itemize}

  \item Kategoria \emph{Natural Language Processing} -- przetwarzanie języka naturalnego, czyli analiza i generowanie tekstu.
        Scenariusze w tej kategorii:

        \begin{itemize}

          \item \emph{Text classification} -- Klasyfikacja tekstu, kategoryzuje surowy tekst wejściowy.
          Scenariusza klasyfikacji tekstu można użyć, jeśli masz zestaw dokumentów lub komentarzy i chcesz sklasyfikować je w różnych kategoriach.

        \end{itemize}

\end{itemize}

W przypadku problemu badawczego tej pracy najlepiej sprawdzi się scenariusz \emph{Image classification}, ponieważ detekcja choroby Alzheimera i stopnia demencji dążyć będzie do zaklasyfikowania obrazów mózgu do jednej z kilku kategorii, w zależności od tego, czy dany obraz jest zdrowego mózgu lub z demencją w jednym z kilku określonych stopni zaawansowania.

Po wybraniu scenariusza testowego Model Builder przekierowuje dalej do wyboru środowiska, na którym uczony będzie model.
W przypadku scenariusza \emph{Image classification} dostępne są trzy opcje: uczenie lokalne na maszynie z użyciem procesora (\emph{Local (CPU)}), uczenie lokalne z użyciem procesora graficznego (\emph{Local (GPU)}) oraz uczenie z wykorzystaniem środowiska chmurowego \emph{Azure}.
Te ostatnie możliwe jest do przeprowadzenia tylko dla scenariuszy klasyfikacji obrazów oraz wykrywania obiektów.
Co ciekawe scenariusz wykrywania obiektów można przeprowadzić wyłącznie w chmurze, lokalne uczenie nie jest możliwe.
Poza nim wszystkie pozostałe scenariusze pozwalają na uczenie przy użyciu lokalnego procesora CPU.
Natomiast na uczenie z użyciem procesora graficznego GPU jest dostępne tylko dla scenariuszy klasyfikacji obrazów oraz klasyfikacji tekstu.
Dodatkowo wymaganiem do jego przeprowadzenia jest zgodność procesora graficznego z technologią CUDA oraz zainstalowany zestaw narzędzi CUDA wersji $10.1$ oraz bibliotekę akceleracji sprzętowego \emph{CUDA Deep Neural Network} (\emph{cuDNN}) w wersji $7.6.4$.
Ważne jest, aby zainstalowana wersja była dokładnie tą sugerowaną -- niższa, a także nawet wyższa poskutkuje brakiem możliwości przeprowadzeni uczenia z użyciem lokalnego procesora graficznego.

W celu przeprowadzenia uczenia lokalnie w jak najkrótszym czasie najlepiej jest wybrać opcję \emph{Local (GPU)}.

Po wybraniu środowiska Model Builder prosi o wybranie zbioru danych uczących.
W przypadku scenariusza \emph{Image classification} Model Builder wymaga podania ścieżki do folderu, w którym znajdują się dane uczące.
W folderze tym powinny znajdować się podfoldery, których nazwy będą odpowiadały klasom / kategoriom, do których należą obrazy w nich zawarte.
W podfolderach tych powinny znajdować się obrazy w formacie \emph{.jpg}, \emph{.png} lub \emph{.bmp}.

Przykład tego, jak wybór zbioru danych testowych wygląda znajduje się na zrzucie ekranu przedstawionym na \hyperref[fig:ml-net-model-builder-2-training-data]{rysunku \ref*{fig:ml-net-model-builder-2-training-data}}.

\begin{figure}[ht]
  \includegraphics[width=\textwidth]{ml-net-model-builder-2-training-data}
  \caption{Zrzut ekranu ze środowiska Visual Studio pokazujący wybór danych uczących w \emph{ML.NET Model Builder}}
  \label{fig:ml-net-model-builder-2-training-data}
\end{figure}

Model Builder po wczytaniu danych przedstawia też wykryte klasy / kategorie, ilość obrazów w każdej nich oraz zsumowanych razem.
Wyświetla także przykładowe 8 z nich, tak, aby użytkownik mógł zweryfikować, czy dane zostały poprawnie wczytane oraz co się w nich znajduje.

Z opcji konfiguracyjnych Model Builder pozwala jeszcze wyłącznie na wybór jednej z czterech metryk -- mikro-dokładność (ang. \emph{micro-accuracy}), makro-dokładność (ang. \emph{macro-accuracy}), strata logarytmiczna (ang. \emph{log-loss}) lub logarytmiczna redukcja strat (ang. \emph{log-loss reduction}).
Więcej opcji konfiguracyjnych uczenie Model Builder nie posiada, użytkownik nie ma kontroli nad liczbą epok, wielkością paczki danych czy współczynnikiem uczenia -- te hiperparametry dobierane są automatycznie przez Model Builder.
Podział danych na zbiór uczący, testowy oraz walidacyjny jest również dokonywany automatycznie.

Bezpośrednio po wybraniu zbioru danych testowych rozpoczyna się uczenie faktyczne, którego postęp jest wyświetlany we wbudowanym terminalu środowiska Visual Studio oraz zapisywany jest również do pliku logów.

Po zakończeniu uczenia Model Builder wyświetla podsumowanie, które zawiera informacje o dokładności wybranej metryki modelu na zbiorze testowym oraz czasie uczenia.
Ekran podsumowujący pozwala też dodać do otwartej solucji przykładowy projekt z wykorzystaniem wytrenowanego modelu, który można wykorzystać do testów lub dalszego rozwoju.
Wygenerowany projekt posiada plik o rozszerzeniu \emph{.mlnet}, który zawiera definicję modelu i sam wytrenowany model.
Pozwala to na łatwe wykorzystanie modelu w innych projektach bez konieczności ponownego przeprowadzania uczenia.
Dodatkowo wygenerowane zostały pliki \emph{.consumption.cs} oraz \emph{.training.cs}, które zawierają kod w języku C\# odpowiednio do konsumpcji modelu oraz do jego ponownego uczenia.
Pozostałe wygenerowane pliki są zgodne z typem wybranej przykładowej aplikacji, na przykład projekt WebApi pozwalający na wykorzystanie modelu w aplikacji REST API lub projekt konsolowy pozwalający na wykorzystanie modelu z poziomu aplikacji konsolowej.

\section{Biblioteka TensorFlow.NET}
\label{sec:tensorflownet}

\emph{TensorFlow.NET} jest biblioteką do uczenia maszynowego i głębokiego uczenia maszynowego dla platformy .NET i języka C\#.
Jest to port biblioteki TensorFlow, która jest jedną z najpopularniejszych bibliotek uczenia maszynowego i głębokiego uczenia maszynowego na świecie omówionej dokładniej w \hyperref[sec:tensorflow-and-keras]{sekcji \ref*{sec:tensorflow-and-keras}}.

Ten port do platformy \emph{.NET} całkowicie omija najpopularniejszy sposób wykorzystywania biblioteki TensorFlow, czyli poprzez język Python.
Niektóre bowiem biblioteki nakładkowe w .NET dające dostęp do TensorFlow wykorzystują Pythona jako warstwę pośrednią, co powoduje konieczność instalacji Pythona i biblioteki TensorFlow wraz z wszystkimi jej zależnościami.
\emph{TensorFlow.NET} natomiast posiada powiązania bezpośrednio do kodu \emph{C} bibliotek TensorFlow, co pozwala na wykorzystanie ich w języku C\# bez konieczności instalacji Pythona oraz z maksymalną możliwą wydajność.

Projekt jest rozwijany przez społeczność programistów i jest dostępny na licencji \emph{Apache 2.0} na portalu GitHub \cite{scisharp-tensorflownet-repo}.
Jest on również częścią szerszego ekosystemu \emph{SciSharp}, który jest zbiorem bibliotek przeznaczonych do zadań z dziedziny Data Science i inżynierii danych dla platformy .NET.
W jego skład wchodzą też porty innych, bardzo popularnych bibliotek języka Python, takie jak NumSharp (port NumPy), Pandas.NET (port Pandas) czy SharpCV (port OpenCV).

W przeciwieństwie do \emph{ML.NET}, \emph{TensorFlow.NET} nie jest frameworkiem uczenia maszynowego, a biblioteką, która pozwala na wykorzystanie nie tylko gotowych modeli uczenia maszynowego i głębokiego uczenia maszynowego, ale także na tworzenie własnych modeli.
Udostępnia ona prawie wszystkie możliwości biblioteki TensorFlow, a co za tym idzie także możliwość jawnego definiowania struktury modelu uczenia maszynowego, wraz jego warstwami i parametrami.
Jest to szczególnie ważne w przypadkach, gdzie problem badawczy jest nietypowy lub nieco trudniejszy, gdzie modele oparte na uczeniu transferowym lub inne uniwersalne i tylko dostrajane rozwiązania mogą nie wystarczyć.

\emph{TensorFlow.NET} stanowi więc bardzo dobry wybór dla programistów, którzy chcą wykorzystać uczenie maszynowe w swoich projektach, ale nie chcą ograniczać się do gotowych rozwiązań, a chcą mieć pełną kontrolę nad tworzonym modelem.
Port oparty bezpośrednio na kodzie C jest bardzo wydajnym rozwiązaniem pozwalającym wykorzystać istniejące a także zbudować nowe modele uczenia maszynowego i głębokiego uczenia maszynowego działające w środowisku \emph{.NET}.
Co więcej możliwe jest użycie tej biblioteki do wytrenowanie wysoce zoptymalizowanego modelu, zapisanie go w dowolnym formacie obsługiwanym także przez TensorFlow (przykładowo \emph{ONNX}, \emph{SavedModel} czy \emph{HDF5}) i wykorzystanie wczytanie go zarówno w innych językach oprogramowania, jak także chociażby przez projekt wykorzystujący ML.NET, który jest nieco lepiej przystosowany do wykorzystania w produkcyjnych aplikacjach.

Spokrewniona biblioteka \emph{TensorFlow.Keras} jest rozszerzeniem biblioteki \emph{TensorFlow.NET}, które pozwala na tworzenie modeli uczenia maszynowego i głębokiego uczenia maszynowego w sposób bardziej wysokopoziomowy, wykorzystując do tego celu warstwy i modele z biblioteki \emph{Keras}.
Podobnie jak \emph{TensorFlow.NET}, \emph{TensorFlow.Keras} jest portem biblioteki \emph{Keras}, która jest jedną z najpopularniejszych bibliotek uczenia maszynowego i głębokiego uczenia maszynowego na świecie omówionej dokładniej w \hyperref[sec:tensorflow-and-keras]{sekcji \ref*{sec:tensorflow-and-keras}}.
Pozwala na prostsze tworzenie modeli, które są bardziej czytelne i łatwiejsze do zrozumienia, a także na wykorzystanie gotowych modeli z biblioteki \emph{Keras}.
Z tego powodu też, podobnie jak \emph{Keras} jest najczęstszym sposobem wykorzystania biblioteki \emph{TensorFlow} w warunkach rzeczywistych, tak \emph{TensorFlow.Keras} jest najczęstszym sposobem wykorzystania biblioteki \emph{TensorFlow.NET}.

\chapter{Uczenie modelu detekcji choroby Alzheimera oraz jego wykorzystanie}

W tej pracy chcę przedstawić możliwości wykorzystania bibliotek uczenia maszynowego w środowisku \emph{.NET} w celu uczenia oraz późniejszego wykorzystania modelu głębokiej sieci neuronowej do detekcji choroby Alzheimera oraz stopnia demencji na obrazach rezonansu magnetycznego mózgów pacjentów, a także postaram się je porównać między sobą oraz z istniejącymi rozwiązaniami także spoza środowiska \emph{.NET}.

\section{Zbiór danych}

Wykorzystany zbiór pochodzi z platformy Kaggle (udostępniony jest pod adresem \url{kaggle.com/datasets/tourist55/alzheimers-dataset-4-class-of-images}, zapewniony na podstawie licencji \emph{Open Data Commons Open Database License} (\emph{ODbL}) wersji $1.0$) \cite{kaggle-alzheimers-dataset}.
Jest wstępnie podzielony na zbiór treningowy oraz testowy w celu zapewnienia powtarzalności i porównywalności wyników uzyskanych z użyciem różnych narzędzi i z różnych źródeł.
Zawiera on zestaw obrazów uzyskanych z badań rezonansem magnetycznym (MRI), których celem jest analiza i diagnoza otępienia spowodowanego chorobą Alzheimera.
Obrazy te są dwuwymiarowym wycinkiem trójwymiarowego skanu rezonansu magnetycznego mózgu, który najlepiej obrazuje strukturę mózgu i potencjalne zmiany chorobowe.
Mają wymiary $208 \times 176$ pikseli -- wystarczająco duże, aby widoczne były nawet drobne detale ale na tyle małe, aby były możliwe do przetworzenia przez mniejsze sieci neuronowe a także umożliwiły ich znacznie szybsze szkolenie.

Zbiór danych składa się z czterech kategorii obrazów, zarówno w zbiorze treningowym, jak i testowym:

\begin{itemize}

  \item
        Brak demencji (\emph{Non Demented}) -- ta kategoria zawiera obrazy mózgów osób niebędących dotkniętymi demencją.
        W zbiorze treningowym znajduje się 2560 obrazów, a w zbiorze testowym 640 obrazów, co daje sumarycznie 3200 zdjęć z tej kategorii.

  \item
        Bardzo łagodna demencja (\emph{Very Mild Demented}) -- zbiór ten obejmuje obrazy osób z bardzo łagodnym stopniem demencji.
        W zbiorze treningowym znajduje się 1792 obrazy, a w zbiorze testowym 448 obrazów, co daje łącznie 2240 obrazów z tej kategorii.

  \item
        Łagodna demencja (\emph{Mild Demented}) -- ten zbiór zawiera obrazy mózgów pacjentów cierpiących na łagodne otępienie związanego z chorobą Alzheimera.
        W zbiorze treningowym znajduje się 717 obrazów, a w zbiorze testowym 179 obrazów, co daje łączną sumę 896 obrazów z tej kategorii.

  \item
        Umiarkowana demencja (\emph{Moderate Demented}) -- ta kategoria obejmuje obrazy mózgów pacjentów już z wyższym stopniem demencji związanym z chorobą Alzheimera.
        W zbiorze treningowym znajduje się 52 obrazy, a w zbiorze testowym 12 obrazów, co daje łącznie 64 obrazy z tej kategorii.

\end{itemize}

\begin{figure}[ht]
  \includegraphics[width=\textwidth]{dataset-image-examples}
  \caption[Zestawienie przykładowych obrazów z każdej kategorii zbioru danych]{Zestawienie przykładowych obrazów z każdej kategorii zbioru danych \cite{kaggle-alzheimers-dataset}, gdzie przedstawione od lewej są: obraz mózgu osoby bez demencji, obraz mózgu osoby z bardzo lekką demencją, obraz mózgu osoby z lekką demencją o podłożach w chorobie Alzheimera oraz obraz mózgu osoby z umiarkowaną demencją chorą na Alzheimera.}
  \label{fig:dataset-image-examples}
\end{figure}

Przykłady obrazów z poszczególnych kategorii przedstawione są na \hyperref[fig:dataset-image-examples]{rysunku \ref*{fig:dataset-image-examples}}.

\section{Uczenie modelu z użyciem narzędzia ML.NET}

\subsection{Uczenie modelu z wykorzystaniem niestandardowego kodu biblioteki ML.NET}

\begin{figure}[ht]
  \includegraphics[width=\textwidth]{plot-mlnet-custom-training-overview}
  \caption[Wykresy statystyk modelu ML.NET Custom w trakcie uczenia]{Wykresy dokładności (\emph{accuracy}) oraz straty (\emph{loss}) dla danych testowych i walidacyjnych modelu ML.NET Custom w trakcie uczenia}
  \label{fig:plot-mlnet-custom-training-overview}
\end{figure}

\subsection{Uczenie modelu z użyciem narzędzia ML.NET Model Builder}

\begin{figure}[ht]
  \includegraphics[width=\textwidth]{plot-mlnet-model-builder-training-overview}
  \caption[Wykresy statystyk modelu ML.NET Model Builder w trakcie uczenia]{Wykresy dokładności (\emph{accuracy}) oraz straty (\emph{loss}) dla danych testowych i walidacyjnych modelu ML.NET Model Builder w trakcie uczenia}
  \label{fig:plot-mlnet-model-builder-training-overview}
\end{figure}

\section{Uczenie niestandardowego modelu z użyciem biblioteki TenserFlow.NET}

\begin{figure}[ht]
  \includegraphics[width=\textwidth]{plot-tensorflownet-training-overview}
  \caption[Wykresy statystyk modelu Tensorflow.NET w trakcie uczenia]{Wykresy dokładności (\emph{accuracy}) oraz straty (\emph{loss}) dla danych testowych i walidacyjnych modelu Tensorflow.NET w trakcie uczenia}
  \label{fig:plot-tensorflownet-training-overview}
\end{figure}

Opis procesu uczenia modelu z użyciem biblioteki TenserFlow.NET

\section{Porównanie wyników}

\begin{figure}[ht]
  \includegraphics[width=\textwidth]{plot-mlnet-custom-vs-mlnet-model-builder}
  \caption[Porównanie dokładności oraz straty modeli ML.NET Custom oraz ML.NET Model Builder]{Porównanie dokładności (\emph{accuracy}) oraz straty (\emph{loss}) na zbiorze treningowym i walidacyjnym modeli z projektów ML.NET Custom oraz ML.NET Model Builder}
  \label{fig:plot-mlnet-custom-vs-mlnet-model-builder}
\end{figure}

\begin{figure}[ht]
  \includegraphics[width=\textwidth]{plot-mlnet-model-builder-vs-tensorflownet}
  \caption[Porównanie dokładności oraz straty modeli ML.NET Model Builder oraz Tensorflow.NET]{Porównanie dokładności (\emph{accuracy}) oraz straty (\emph{loss}) na zbiorze treningowym i walidacyjnym modeli z projektów ML.NET Model Builder oraz Tenserflow.NET}
  \label{fig:plot-mlnet-model-builder-vs-tensorflownet}
\end{figure}

\begin{table}[ht]
  \centering
  \begin{tabular}{|l|r|r|r|r|}
    \hline
                         & \multicolumn{2}{c|}{Zbiór treningowy}                                          & \multicolumn{2}{c|}{Zbiór walidacyjny}                      \\
    \cline{2-5}
                         & \multicolumn{1}{|c|}{Accuracy}        & \multicolumn{1}{|c|}{Loss}             & \multicolumn{1}{|c|}{Accuracy} & \multicolumn{1}{|c|}{Loss} \\
    \hline
    ML.NET Custom        & 0.7373874                             & 0.6378530                              & 0.7359702                      & 0.6511808                  \\
    ML.NET Model Builder & 0.7401715                             & 0.6036751                              & 0.6793103                      & 0.7503802                  \\
    Tensorflow.NET       & 0.9118868                             & 0.8311445                              & 0.8935547                      & 0.8501284                  \\
    \hline
  \end{tabular}
  \caption[Porównanie dokładności oraz straty modeli na zbiorze treningowym i walidacyjnym]{Porównanie dokładności (\emph{accuracy}) oraz straty (\emph{loss}) modeli na zbiorze treningowym i walidacyjnym w epokach szkolenia najlepszych względem na dokładności walidacyjnej}
  \label{tab:train_validation_metric_comparison}
\end{table}

\begin{table}[ht]
  \centering
  \begin{tabular}{|l|r|}
    \hline
                         & \multicolumn{1}{|c|}{Accuracy} \\
    \hline
    ML.NET Custom        & 0.45269742                     \\
    ML.NET Model Builder & 0.70602033                     \\
    Tensorflow.NET       & 0.85926505                     \\
    \hline
  \end{tabular}
  \caption[Porównanie dokładności modeli na tym samym zbiorze testowym]{Porównanie dokładności (\emph{accuracy}) modeli na tym samym zbiorze testowym}
  \label{tab:test_accuracy_comparison}
\end{table}

\section{Wykorzystanie modelu w aplikacji z użyciem biblioteki ML.NET}

W jaki sposób wykorzystać można model w aplikacji

\chapter{Dyskusja}

Wykorzystanie uczenia maszynowego w celach diagnostyki chorób nie jest nowym pomysłem.
Tego typu zastosowania pojawiały się już także w kontekście chorób neurodegeneracyjnych, w tym choroby Alzheimera.

Autorzy jednej z prac wykorzystali technikę uczenia maszynowego nazywaną maszyną wektorów nośnych (ang. \emph{Support Vector Machine}, SVM), czyli tej najczęściej pojawiajacej się w literaturze, jak zostało opisane w \hyperref[modern-detection-methods-for-alzheimers-using-machine-learning]{sekcji \ref*{modern-detection-methods-for-alzheimers-using-machine-learning}} \cite{trambaiolli2011improving}.
W badaniu tym wykorzystano dane encefalograficzne (EEG) i zastosowano metodę analizy ilościowej EEG (qEEG) do automatycznego rozróżniania pacjentów z chorobą Alzheimera od osób zdrowych, jako uzupełnienie diagnozy prawdopodobnej demencji.
Autorom udało się przy tym podejściu uzyskać dokładność na poziomie $79.9\%$ bazując wyłącznie na danych EEG, oraz $87.0\%$ przy uwzględnieniu diagnozy każdego indywidualnego pacjenta.
Porównując moje rezultaty z tymi opisywanymi, modele uczenia maszynowego powstałe w ramach tej pracy osiągnęły wyższą dokładność klasyfikacji bazując wyłącznie na obrazach medycznych.
Jednakże warto mieć tutaj na uwadze, że autorzy przeprowadzili te badania w roku 2011 kiedy technologie uczenia głębokiego nie były jeszcze tak rozwinięte jak obecnie.

Nowsze prace wykorzystujące uczenie głębokie w celu diagnozy choroby Alzheimera osiągają znacznie lepsze wyniki.
W szczególności sprawdzają się rozwiązania w postaci głębokich konwolucyjnych sieci neuronowych.
Autorzy jednej z takich prac z roku 2018 osiągnęli dokładność na poziomie $88.24\%$ \cite{shahbaz2019classification}, a więc bardzo zbliżoną do najlepszego wyniku osiągniętego przeze mnie.
W innej pracy z roku 2021 autorom udało się osiągnąć dokładność na poziomie $92.0\%$ \cite{ebrahimi2021deep}, a więc wyższą niż najlepszy osiągnięty przeze mnie wynik o około 6 punktów procentowych.

Co ciekawe, autorzy drugiej z tych prac przeanalizowali także użycie uczenia transferowanego do rozwiązania tego samego problemu.
Według ich podsumowania 3 modele detekcji choroby Alzheimera, które pokryły się z tymi dostępnymi w \emph{ML.NET}, a więc \lstinline{InceptionV3}, \lstinline{ResnetV2101} oraz \lstinline{ResnetV250} osiągnęły dokładność klasyfikacji odpowiednio $78\%$, $79\%$ oraz $81\%$.
Są to wszystko wartości wyższe niż najlepszy wynik modelu bazującego na uczeniu transferowanym \emph{ML.NET Model Builder}, który osiągnął dokładność na poziomie $71\%$.
Może to sugerować, iż biblioteka \emph{ML.NET} nie pozwala na optymalne wykorzystanie wstępnie wytrenowanych modeli.

Jedne z najnowszych prac badawczych nad wykorzystaniem uczenia maszynowego w celach diagnostyki choroby Alzheimera osiągają wyniki jeszcze lepsze.
Autorzy pracy z roku 2022 przeanalizowali kilka metod uczenia maszynowego w celu rozpoznania choroby Alzheimera na podstawie obrazów MRI, w tym niestandardową głęboką konwolucyjną sieć neuronową oraz uczenie transferowe bazowane na wstępnie wytrenowanym modelu \lstinline{ResNet101} \cite{mamun2022deep}.
Ich praca więc jest bardzo zbliżona do mojego podejścia z wyłączeniem używanych technologii.
Dla sieci konwolucyjnej udało się osiągnąć dokładność aż $97.6\%$, znacznie wyższą niż najlepszy wynik osiągnięty przeze mnie.
Był to najlepszy wynik osiągnięty także przez autorów, którzy otrzymali wynik podobny do mojego, gdzie niestandardowa implementacja głębokiej konwolucyjnej sieci neuronowej radziła sobie lepiej niż uczenie transferowe.

Ważnym do zauważenia szczegółem jest fakt, że wymienione badania skupiały się na binarnej detekcji choroby Alzheimera, gdzie w mojej pracy modele były trenowane także do stopniowania demencji na trzy dodatkowe kategorie przewlekłości, co utrudnia przestrzeń problemową.

Istotnym szczegółem wartym są użyte technologie.
Celem mojej pracy było wykorzystanie uczenia maszynowego do detekcji choroby Alzheimera oraz stopnia demencji z użyciem narzędzi uczenia maszynowego dostępnych w środowisku \emph{.NET}.
Nie udało mi się znaleźć \emph{żadnej} pracy, która próbowałaby rozwiązać ten problem z wykorzystaniem technologii \emph{.NET}, co ze wzgląd na brak popularności tego środowiska nie jest zaskakujące.
Co więcej dostępne narzędzia uczenia maszynowego są gorsze i często bazują na odpowiednikach z innych technologii, jednak przez brak wsparcia społeczności ich rozwój jest ograniczony.

Ograniczony rozwój i niedojrzałe narzędzia były odczuwalne szczególnie w próbie uczenia sieci neuronowej z użyciem \emph{TensorFlow.NET}, gdzie brakowało dostępnych istotnych funkcjonalności.
Przede wszystkim chodzi o brak dostępnych metryk poza standardową metryką dokładności (\lstinline{acc}), która dla dostępnych danych nie była optymalnym wyborem.
Zbiór danych bowiem był niezbalansowany i różne klasy posiadały znacznie różną liczbę przykładów.
Dla tego typu danych najlepszym rozwiązaniem jest metryka ROC AUC (\lstinline{auc}), która jest dostępna w \emph{TensorFlow}, ale nie w \emph{TensorFlow.NET}.
Należy zaznaczyć że mimo aktualnej niedojrzałości narzędzi są one aktywnie rozwijane, czego przykładem mógł być brak funkcji straty \lstinline{CategoricalCrossentropy} w momencie rozpoczynania pisania kodu projektów, która w jego trakcie została dodana i umożliwiła jego wykorzystanie.

Ta praca przeanalizowała największe i najbardziej popularne narzędzia uczenia maszynowego dostępne w środowisku \emph{.NET}, jednak nie są one wszystkimi.
Potencjalnym kierunkiem dalszych prac mogłoby być przeanalizowanie dodatkowych bibliotek takich jak \emph{Accord.NET Machine Learning Framework} lub \emph{TorchSharp}, które mogą potencjalnie uzyskiwać lepsze wyniki.
Warto równie powrócić do tematu bibliotek już przeze mnie opisanych w tej pracy z czasem, gdyż ich rozwój może przynieść nowe funkcjonalności i poprawić wyniki.
\emph{ML.NET} może potencjalnie uzyskać możliwość wykorzystania uczenia transferowego z użyciem jeszcze nowszych, większych i skuteczniejszych modeli, takich jak \emph{VGG-19}, które w niektórych pracach osiąga bardzo wysokie wyniki
rzędu $83.72\%$ dla klasyfikacji stopnia demencji oraz aż $98.73\%$ przy samej binarnej detekcji choroby Alzheimera \cite{mehmood2021transfer}.
Podobnie rozwój \emph{TensorFlow.NET} może przynieść brakujące funkcjonalności, które pozwolą na uzyskanie lepszych wyników.

Mimo potencjału rozwoju środowisko \emph{.NET} nie jest optymalne w kwestii przeprowadzania uczenia maszynowego.
Istnieją natomiast narzędzia, które pozwolą na wykorzystanie modelu wytrenowanego z pomocą innych narzędzi i środowisk w rozwiązaniach \emph{.NET}, co umożliwia kompromis między skutecznością uczenia oraz możliwościami jego wykorzystania w praktyce.


% Add the Bibliography to the contents page
\addcontentsline{toc}{chapter}{Bibliografia}
% Use a bibtex bibliography file refs.bib
\bibliography{refs}
% Use the plain bibliography style ordered by citation
\bibliographystyle{unsrt}

\end{document}
