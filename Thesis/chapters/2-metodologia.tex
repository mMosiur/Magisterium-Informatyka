\chapter{Metodologia i narzędzia uczenia maszynowego środowiska .NET}

Celem tej pracy jest wykorzystanie technologii uczenia maszynowego w środowisku .NET w celu detekcji choroby Alzheimera.
W tym rozdziale więc zostaną przedstawione narzędzia, które pomogą w osiągnięciu tego celu, w tym technologie uczenia maszynowego oraz sama platforma .NET.

\section{Technologie głębokiego uczenia maszynowego}

Teoretyczne zagadnienia związanie z technologiami głębokiego uczenia maszynowego, w szczególności głębokich konwolucyjnych sieci neuronowych zostały już omówione w sekcji \ref{sec:deep-learning}.
W tym rozdziale zostanie przedstawione w jaki sposób obiecująca teoria jest przekładana na wykorzystanie w praktyce.

\subsection{Symulator SNNS}

Tematykę oprogramowania służącego do tworzenia i trenowania sieci neuronowych warto zacząć omówienia przestarzałego już programu \emph{SNNS} (ang. Stuttgart Neural Network Simulator).
Pozwala on na tworzenie i trenowanie sieci neuronowych, a także -- co ważne przy jego zastosowaniach -- na ich wizualizację.

SNNS to program okienkowy napisany w języku C, przeznaczony głównie dla systemów Unixowych.
W ramach jego działania można zamodelować sieć neuronową po jednym neuronie, łącząc je z dużą dozą swobody w dowolne struktury oraz parametryzując je w dowolny sposób.
Następnie można zdefiniować zbiór danych uczących, a także zbiór danych testowych, na których można przeprowadzić proces uczenia sieci.

Całe tworzenie sieci jest czynnością bardzo czasochłonną i złożoną ze względu na manualną naturę definiowania jej struktury.
Daje za to bardzo dużo możliwości modyfikacji, w tym zmianę na przykład funkcji aktywacji pojedynczego neuronu, zachowania się konkretnego połączenia oraz wielu innych właściwości, które posiadają wszystkie obiekty możliwe do edycji w programie.

Najważniejszą cechą SNNS jest wizualny sposób budowy sieci neuronowej oraz manualny proces jej konfiguracji.
Nie jest ona może wtedy wystarczająco wydajna do jakichkolwiek problemów, z którymi potrafią radzić sobie nowoczesne sieci neuronowe, ale pozwala na zrozumienie ich działania oraz nauczenie się podstawowych zasad ich budowy.
Dlatego też mimo, iż sam program jest już przestarzały i wycofany z użytku a sami jego autorzy polecają użycie nowoczesnych bibliotek uczenia maszynowego takich jak TensorFlow czy PyTorch \cite{snns}, to nadal jest bardzo często wykorzystywany w celach edukacyjnych.

Pomimo, iż znane są metody znacznie przyspieszające działanie i uczenie sieci neuronowych przez reprezentacje wag i aktywacji jako macierzy i wektorów i ich późniejsze mnożenie przy pomocy zrównoleglonych obliczeń na kartach graficznych i dedykowanych urządzeniach, to założenia podstaw działania sztucznych sieci neuronowych pozostają niezmienne i programy typu SNNS pozwalają na znacznie prostsze ich poznanie i zrozumienie.

\subsection{Azure Machine Learning Studio}

\emph{Microsoft Azure Machine Learning Studio} (w skrócie \emph{AML Studio}) to kompleksowa platforma, która wykorzystywana jest do implementacji i zarządzania procesami uczenia maszynowego.
AML Studio oferuje zaawansowane narzędzia do tworzenia, wdrażania oraz monitorowania modeli uczenia maszynowego, integrując różnorodne etapy tego procesu w jednym środowisku.

Jest to platforma oparta na chmurze i posiadająca interfejs graficzny w postaci intuicyjnej strony internetowej, na której modelowanie przetwarzania danych i procesu uczenia odbywa się przy użyciu przeciągania i upuszczania elementów blokowych oraz łączenia ich w odpowiedni sposób.
Pozwala na tworzenie tak zwanych \emph{eksperymentów}, które składają się z kolejnych kroków w analizie danych i tworzeniu modeli.
Dzięki temu możliwe jest systematyczne badanie różnych podejść i łatwe porównywanie ich wyników.
Graficzny sposób reprezentacji wykorzystanych modułów w eksperymencie a także przepływu danych między nimi pozwala na uproszczenie wyszukiwania potencjalnych błędów lub problemów pojawiających się w całym procesie.
Możliwe jest również analizowanie w czasie rzeczywistym działania modelu i analizę przez niego danych w całym eksperymencie.

Bardzo istotną cechą AML Studio jest fakt, że platforma integruje się z innymi usługami chmurowymi dostarczanymi przez Microsoft Azure, co umożliwia tworzenie spójnych rozwiązań opartych na chmurze.
Pozwala na dynamiczne wczytywanie danych z innych źródeł znajdujących się w chmurze, a także na wdrażanie modeli uczenia maszynowego w postaci usług sieci Web, które mogą być wykorzystywane przez inne aplikacje.

Wykorzystanie Azure Machine Learning Studio dzięki swoim cechom pozwala na szybki trening modelu uczenia maszynowego oraz jego wdrożenie bez konieczności pisania kodu czy głębszej znajomości tematyki uczenia maszynowego \cite{mukunthu2019practical}, w szczególności gdy z założenia wdrożony model ma działać w chmurze i integrować się z innymi systemami w niej obecnymi.
Jednak w zastosowaniach, które nie czerpią korzyści z połączenia z chmurą i dają swobodę czasową na własnoręczne zaimplementowanie modelu, wykorzystanie AML Studio może być nieopłacalne ze względu na koszty związane z jego wykorzystaniem w większych ilościach.
Warto wtedy rozważyć inne rozwiązania, które dają swobodę konstruowania i trenowania modelu od podstawi i dostrojenie go do bardziej złożonych zadań, maksymalizując tym samym osiągane możliwości i dokładność -- a są to zazwyczaj biblioteki uczenia maszynowego w językach programowania.

\subsection{Tensorflow i Keras}

Jedną z najpopularniejszych bibliotek uczenia maszynowego, która jest wykorzystywana do tworzenia i trenowania sieci neuronowych jest \emph{Tensorflow}.
Jest to otwartoźródłowa platforma do obliczeń numerycznych i implementacji modeli uczenia maszynowego.
Jego fundamentem jest reprezentacja danych w postaci tensorów, co umożliwia wykonywanie skomplikowanych operacji matematycznych na dużych zbiorach danych \cite{shukla2018machine}.

Tensorflow jest bilbioteką niskopoziomową napisaną głównie w języku C++ i implementującą często używane algorytmy uczenia maszynowego i sieci neuronowych i wystawiającą je jako interfejs programistyczny w językach wyższego poziomu takich jak Python czy JavaScript.
Napisany został pierwotnie przez zespół badawczy Google Brain.

Jednak ze względu swoje dążenie do umożliwienia jak największej elastyczności i kontroli nad procesem uczenia sieci neuronowych, Tensorflow w swojej czystej postaci wymaga od programisty dużo pracy i wiedzy, aby zaimplementować nawet najprostszy model sieci neuronowej.

W celu uproszczenia z korzystania z Tensorflow powstały ``frontendy'' dla różnych języków programowania pozwalające na wykorzystanie jego możliwości w bardziej przyjazny sposób.
Jednym z nich jest \emph{Keras}, który jest wysokopoziomowym interfejsem programistycznym dla Tensorflow w języku Python.
Jest to jeden z najpopularniejszych interfejsów programistycznych dla Tensorflow ze względu na swoją prostotę i intuicyjność.

Dla uproszczenia wystawia API sekwencyjne (ang. Sequential API), które przez wykorzystanie istniejących klas i ``sekwencyjne'' tworzenie obiektów symbolizujących określone rodzaje warstw ze zdefiniowanymi odpowiednio parametrami pozwala na proste w śledzeniu i zrozumieniu stworzenie struktury sieci neuronowych.
Operacje takie jak \emph{skompilowanie} (czyli utworzenie z definicji warstw sieci zoptymalizowanego do uruchomienia i uczenia wstępnego modelu) definicji modelu używając odpowiednich parametrów czy uruchomienie trenowania (ang. \emph{fit}) w odpowiedniej konfiguracji wykonuje się wywołaniem tylko  jednej metody.

Mimo swojej prostoty, Keras pozwala na również na podejścia bardziej eksperckie.
W nich oprócz definiowania sekwencyjnego struktury sieci można również tworzyć własne klasy reprezentujące niestandardowe warstwy lub całe bloki odpowiednio ustrukturyzowane i zparametryzowane.

Uczenie sieci neuronowej w Keras zachowuje również zbiór metryk, które pozwalają na prześledzenie postępów w procesie uczenia i wykorzystanie ich do analizy lub optymalizacji modelu albo całego schematu programu.

Sukces Tensorflow oraz Keras wynika głównie z połączenia ich wydajności, elastyczności i prostoty użycia, które w dużej mierze są zasługą również wykorzystanego języka programowania Python, najpopularniejszego do zastosowań w uczeniu maszynownym.
Jednak ze względu na to, że obie te biblioteki mają swoje źródła dostępne na bazie licencji Apache 2.0 w publicznie dostępnym repozytorium serwisu GitHub \cite{tensorflow}.
Pozwala to społeczności na własne próby rozszerzenia zasięgu platformy do innych języków programowania, na przykład jak zostanie opisane w rozdziale \ref{sec:tensorflownet} -- do języka C\#.

\section{Platforma .NET}

Platforma .NET

\section{Framework uczenia maszynowego ML.NET}

\subsection{ML.NET Model Builder}

\subsection{Ręczne tworzenie modelu z użyciem ML.NET}

\label{sec:tensorflownet}
\section{Biblioteka Tensorflow.NET}

Tensorflow.NET
