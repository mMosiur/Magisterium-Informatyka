\chapter{Przegląd piśmiennictwa}

Choroba Alzheimera nadal nie jest w pełni zrozumiana, jednak dzięki badaniom zdobyto już wiele istotnych informacji na jej temat.
Część z nich opiera się wyłącznie na korelacjach i potwierdzonych powiązaniach z pominięciem przyczyn i konkretnych mechanizmów nimi rządzących, lecz nadal daje nam to wystarczający obraz by przedstawić kompleksowo opis choroby.

\section{Historia, definicja i objawy choroby Alzheimera}

\subsection{Wczesna historia}

Istnienie demencji było znane ludzkości od bardzo dawna.
Pierwsze wzmianki na ten temat pojawiały się już w czasach starożytnej Grecji, a informacje dotyczące jej objawów można odnaleźć nawet wcześniej, w starożytnym Egipcie \cite{boller1998history}.

Natomiast choroba Alzheimera została po raz pierwszy opisana przez niemieckiego lekarza Aloisa Alzheimera, od którego nazwiska została później nazwana.
W roku 1901 neuropatolog zajął się przypadkiem Auguste Deter, pacjentki przyjętej do szpitala psychiatrycznego.
Od jej 51. roku życia zaczęły się stopniowe zmiany osobowości, które następnie przerodziły się w postępujące zaburzenia kognitywne, a ostatecznie w całkowitą apatię \cite{cipriani2011alzheimer}.
W momencie, gdy Alzheimer rozpoczął badania nad nią, pacjentka miała już prawie pięcioletnią historię narastających halucynacji, urojeń, apraksji, zaburzeń pamięci, mowy, społecznych i behawioralnych.

Po śmierci pacjentki lekarz przeprowadził badania mózgu zmarłej, w trakcie których znalazł zmiany i cechy dziś przypisywane omawianej chorobie.
Alzheimer przedstawił swoje kliniczne i patologiczne wyniki na konferencji południowo-zachodnich niemieckich psychiatrów w 1906 roku, następnie opublikował w formie pisemnej w czasopismach medycznych poświęconych psychiatrii \cite{alzheimer1906uber}.

W tym czasie jeszcze sam Alois Alzheimer nie był pewien, czy przypadek Auguste Deter reprezentował nowy zespół chorobowy.
Uważał, że opisał bardzo rzadką przypadłość i najprawdopodobniej wierzył, że była to po prostu szczególna forma demencji starczej \cite{cipriani2011alzheimer}.
Dopiero kilka lat później, po raz pierwszy użyty został termin ``choroba Alzheimera'' (\emph{ang. ``Alzheimer's disease'', często po prostu ``AD''}) w wydanym podręczniku psychiatrii Emila Kraeplina \cite{kraepelin1910psychiatrie}.

Jeszcze w pierwszej połowie XX wieku publikacje podkreślały brak zależności przebiegu choroby Alzheimera od wieku pacjentów.
Pojawiła się również istotna kliniczna debata wokół pytania czy demencja ``przedstarcza'' oraz demencja ``starcza'' (\emph{ang. kolejno ``presenile'' oraz ``senile''}) są tym samym zaburzeniem \cite{jellinger2006alzheimer}.
W połowie wieku pojawiły się pierwsze postulaty mówiące, że (przedstarcze) \emph{AD} oraz demencja starcza są całkowicie odrębnymi jednostkami chorobowymi (choć nowsze badania dojść jednoznacznie pokazują że tak nie jest).

W latach 60 XX wieku rozpoczęła się nowa era badań bazująca na nowym, bardzo ważnym narzędziu badawczym -- mikroskopii elektronowej.
Dzięki niej możliwe było zbadanie struktury komórek nerwowych i ich połączeń, a także zbadanie struktury i składu splotów nerwowych znajdujących się w mózgu.
Takich opisów jako jeden z pierwszych dostarczył w 1963 roku Kidd, który opisał zmiany w strukturze splątków neurofibrylarnych oraz blaszek amyloidowych \cite{kidd1963paired}.

Od tego czasu choroba progresywnie zyskiwała coraz większą uwagę ze strony środowiska naukowego, a także społeczeństwa.
W 1974 roku rząd Stanów Zjednoczonych Ameryki powołał do istnienia Narodowy Instytut ds. Starzenia się (\emph{ang. National Institute on Aging, NIA}), który od tego czasu finansuje badania nad chorobą Alzheimera \cite{marx1974aging}.
W 1980 roku powstało ``Stowarzyszenie Choroby Alzheimera i Pokrewnych Zaburzeń'' (\emph{ang. Alzheimer's Disease and Related Disorders Association, ADRA}), organizacja non-profit, która dziś znana jest jako ``Stowarzyszenie Choroby Alzheimera'' (\emph{ang. Alzheimer's Association, AA}).
Wraz z rosnącą świadomością społeczną wzrastało także finansowanie, dzięki czemu choroba jeszcze 30 lat temu uznawana za całkiem nieuleczalną i nie do powstrzymania czy spowolnienia, dziś jest jedną z najbardziej intensywnie badanych chorób neurodegeneracyjnych z prospektami na walkę z nią w niedalekiej przyszłości.

Mimo tych wielu obecnych badań, nadal pozostaje wiele czynników wciąż niepoznanych i nie ma jednoznacznej odpowiedzi na pytanie co jest przyczyną choroby Alzheimera.
Wciąż brakuje również skutecznych metod leczenia, które mogłyby zatrzymać jej postęp lub spowolnić go w znaczący sposób.

Zaraz obok leczenia innym bardzo istotnym aspektem jest wczesna diagnoza.

\section{Diagnoza i stadium demencji}

Diagnoza i stadium demencji

\section{Metody tradycyjne detekcji choroby Alzheimera}

Metody tradycyjne detekcji choroby Alzheimera

\section{Narzędzia i techniki uczenia maszynowego}

Narzędzia i techniki uczenia maszynowego

\section{Istniejące podejścia do detekcji choroby Alzheimera z użyciem uczenia maszynowego}

Istniejące podejścia do detekcji choroby Alzheimera z użyciem uczenia maszynowego
