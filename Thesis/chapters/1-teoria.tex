\chapter{Przegląd piśmiennictwa}

Choroba Alzheimera nadal nie jest w pełni zrozumiana, jednak dzięki badaniom zdobyto już wiele istotnych informacji na jej temat.
Część z nich opiera się wyłącznie na korelacjach i potwierdzonych powiązaniach z pominięciem przyczyn i konkretnych mechanizmów nimi rządzących, lecz nadal daje nam to wystarczający obraz by przedstawić kompleksowo opis choroby.

\section{Historia, definicja i objawy choroby Alzheimera}

\subsection{Historia}

Istnienie demencji było znane ludzkości od bardzo dawna.
Pierwsze wzmianki na ten temat pojawiały się już w czasach starożytnej Grecji, a informacje dotyczące jej objawów można odnaleźć nawet wcześniej, w starożytnym Egipcie \cite{boller1998history}.

Natomiast choroba Alzheimera została po raz pierwszy opisana przez niemieckiego lekarza Aloisa Alzheimera, od którego nazwiska ją później nazwano.
W roku 1901 neuropatolog zajął się przypadkiem Frau Deter, pacjentki przyjętej do szpitala psychiatrycznego.
Od jej 51. roku życia zaczęły się stopniowe zmiany osobowości, które następnie przerodziły się w postępujące zaburzenia kognitywne, a ostatecznie w całkowitą apatię \cite{cipriani2011alzheimer}.

Po śmierci pacjentki lekarz przeprowadził badania mózgu zmarłej, w trakcie których znalazł zmiany i cechy dzisiaj przypisywane omawianej chorobie.
Alzheimer przedstawił swoje kliniczne i patologiczne wyniki na konferencji południowo-zachodnich niemieckich psychiatrów w 1906 roku, następnie opublikował w formie pisemnej w czasopismach medycznych poświęconych psychiatrii.

\section{Diagnoza i stadium demencji}

Diagnoza i stadium demencji

\section{Metody tradycyjne detekcji choroby Alzheimera}

Metody tradycyjne detekcji choroby Alzheimera

\section{Narzędzia i techniki uczenia maszynowego}

Narzędzia i techniki uczenia maszynowego

\section{Istniejące podejścia do detekcji choroby Alzheimera z użyciem uczenia maszynowego}

Istniejące podejścia do detekcji choroby Alzheimera z użyciem uczenia maszynowego
