\chapter{Przegląd piśmiennictwa}

Choroba Alzheimera nadal nie jest w pełni zrozumiana, jednak dzięki badaniom zdobyto już wiele istotnych informacji na jej temat.
Część z nich opiera się wyłącznie na korelacjach i potwierdzonych powiązaniach z pominięciem przyczyn i konkretnych mechanizmów nimi rządzących, lecz nadal daje nam to wystarczający obraz by przedstawić kompleksowo opis choroby.

\section{Historia, definicja i objawy choroby Alzheimera}

\subsection{Wczesna historia}

Istnienie demencji było znane ludzkości od bardzo dawna.
Pierwsze wzmianki na ten temat pojawiały się już w czasach starożytnej Grecji, a informacje dotyczące jej objawów można odnaleźć nawet wcześniej, w starożytnym Egipcie \cite{boller1998history}.

Natomiast choroba Alzheimera została po raz pierwszy opisana przez niemieckiego lekarza Aloisa Alzheimera, od którego nazwiska została później nazwana.
W roku 1901 neuropatolog zajął się przypadkiem Auguste Deter, pacjentki przyjętej do szpitala psychiatrycznego.
Od jej 51. roku życia zaczęły się stopniowe zmiany osobowości, które następnie przerodziły się w postępujące zaburzenia kognitywne, a ostatecznie w całkowitą apatię \cite{cipriani2011alzheimer}.
W momencie, gdy Alzheimer rozpoczął badania nad nią, pacjentka miała już prawie pięcioletnią historię narastających halucynacji, urojeń, apraksji, zaburzeń pamięci, mowy, społecznych i behawioralnych.

Po śmierci pacjentki lekarz przeprowadził badania mózgu zmarłej, w trakcie których znalazł zmiany i cechy dziś przypisywane omawianej chorobie.
Alzheimer przedstawił swoje kliniczne i patologiczne wyniki na konferencji południowo-zachodnich niemieckich psychiatrów w 1906 roku, następnie opublikował w formie pisemnej w czasopismach medycznych poświęconych psychiatrii \cite{alzheimer1906uber}.

W tym czasie jeszcze sam Alois Alzheimer nie był pewien, czy przypadek Auguste Deter reprezentował nowy zespół chorobowy.
Uważał, że opisał bardzo rzadką przypadłość i najprawdopodobniej wierzył, że była to po prostu szczególna forma demencji starczej \cite{cipriani2011alzheimer}.
Dopiero kilka lat później, po raz pierwszy użyty został termin ``choroba Alzheimera'' (\emph{ang. ``Alzheimer's disease'', często po prostu ``AD''}) w wydanym podręczniku psychiatrii Emila Kraeplina \cite{kraepelin1910psychiatrie}.

Jeszcze w pierwszej połowie XX wieku publikacje podkreślały brak zależności przebiegu choroby Alzheimera od wieku pacjentów.
Pojawiła się również istotna kliniczna debata wokół pytania czy demencja ``przedstarcza'' oraz demencja ``starcza'' (\emph{ang. kolejno ``presenile'' oraz ``senile''}) są tym samym zaburzeniem \cite{jellinger2006alzheimer}.
W połowie wieku pojawiły się pierwsze postulaty mówiące, że (przedstarcze) \emph{AD} oraz demencja starcza są całkowicie odrębnymi jednostkami chorobowymi (choć nowsze badania dojść jednoznacznie pokazują że tak nie jest).

W latach 60 XX wieku rozpoczęła się nowa era badań bazująca na nowym, bardzo ważnym narzędziu badawczym -- mikroskopii elektronowej.
Dzięki niej możliwe było zbadanie struktury komórek nerwowych i ich połączeń, a także zbadanie struktury i składu splotów nerwowych znajdujących się w mózgu.
Takich opisów jako jeden z pierwszych dostarczył w 1963 roku Kidd, który opisał zmiany w strukturze splątków neurofibrylarnych oraz blaszek amyloidowych \cite{kidd1963paired}.

Od tego czasu choroba progresywnie zyskiwała coraz większą uwagę ze strony środowiska naukowego, a także społeczeństwa.
W 1974 roku rząd Stanów Zjednoczonych Ameryki powołał do istnienia Narodowy Instytut ds. Starzenia się (\emph{ang. National Institute on Aging, NIA}), który od tego czasu finansuje badania nad chorobą Alzheimera \cite{marx1974aging}.
W 1980 roku powstało ``Stowarzyszenie Choroby Alzheimera i Pokrewnych Zaburzeń'' (\emph{ang. Alzheimer's Disease and Related Disorders Association, ADRDA}), organizacja non-profit, która dziś znana jest jako ``Stowarzyszenie Choroby Alzheimera'' (\emph{ang. Alzheimer's Association, AA}).
Wraz z rosnącą świadomością społeczną wzrastało także finansowanie, dzięki czemu choroba jeszcze 30 lat temu uznawana za całkiem nieuleczalną i nie do powstrzymania czy spowolnienia, dziś jest jedną z najbardziej intensywnie badanych chorób neurodegeneracyjnych z prospektami na walkę z nią w niedalekiej przyszłości.

Mimo tych wielu obecnych badań, nadal pozostaje wiele czynników wciąż niepoznanych i nie ma jednoznacznej odpowiedzi na pytanie co jest przyczyną choroby Alzheimera.
Wciąż brakuje również skutecznych metod leczenia, które mogłyby zatrzymać jej postęp lub spowolnić go w znaczący sposób.

Zaraz obok leczenia innym bardzo istotnym aspektem jest wczesna diagnoza.

\section{Diagnoza i stadium demencji}



\section{Metody diagnostyki choroby Alzheimera}

Między rokiem 1960 a 1985 wprowadzono kilka różnych kategorii obiektywnych klinicznie narzędzi diagnostyki choroby Alzheimera, wiele z nich potwierdzonych w autopsji \cite{khachaturian2006diagnosis}.
Były to między innymi:

\begin{itemize}

  \item \emph{Mini-Mental State Examination} (w skrócie \emph{MMSE}, potocznie mini-mental) -- jest to zaprojektowany w 1975 roku test oceny demencji oraz jej stopnia w formie szybkiego badania, najczęściej krótszego niż 5 minut.
        Mini-mental jest narzędziem przesiewowym (ang. \emph{screening}), to jest -- nie jest dokładne i jednoznaczne, ale jego negatywny wynik daje podstawy przypuszczać prawdopodobną utratę zdolności kognitywnych i wszcząć
        głębsze, kliniczne badania w celu potwierdzenia wstępnej diagnozy.

        W wyniku tego badania uzyskuje się miedzy 0 a 30 punktów, gdzie wraz z malejącą zdobytą ilością rośnie domniemany stopień demencji.
        27 punktów oraz powyżej oznacza wynik prawidłowy, 24 i w górę to niedemenecyjne zaburzenia poznawcze, 19 wzwyż sugerują demencję łagodną, od 11 do 18 demencję średniego stopnia, a poniżej 10 demencję głęboką.

        Niedawne badania nad skutecznością MMSE na przestrzeni różnych kategorii stwierdzały jej czułość (sensitivity) między 23\% a 76\%, a jej specyficzność (specificity) między 40\% a 90\% \cite{arevalo2015mini}.
        Nie jest to więc wynik wysoki, jednakże wciąż jest to istotne narzędzie używane w praktyce klinicznej, głównie ze względu na jego prostotę i szybkość.

  \item \emph{Short Blessed Test} (w skrócie \emph{SBT}) -- badanie opublikowane do użytku w 1983 roku skupiające się na orientacji, pamięci i koncentracji \cite{katzman1983validation}.
        Według samych autorów przeznaczone do wykrywania wczesnych zaburzeń poznawczych, nie samej diagnozy choroby Alzheimera.
        Wszelkie wyniki sugerujące demencję wymagają dalszych badań.

  \item \emph{Clinician's Interview-Based Impression of Change plus caregiver input} (w skrócie \emph{CIBIC+}) -- zaproponowane i wymagane przez amerykańską Agencję Leków i Żywności (ang. \emph{FDA}, \emph{Food and Drug Administration}) podejście do przeprowadzania badań nad lekami na chorobę Alzheimera i demencję, który ma pozwalać na uzyskiwanie mierzalnych klinicznie danych potrzebnych do oceny skuteczności badanych środków \cite{joffres2000qualitative}.
        Najczęściej praktykowanym testem implementującym założenia CIBIC+ jest \emph{Alzheimer's Disease Cooperative Study -- Clinical Global Impression of Change} (w skrócie \emph{ADCS-CGIC}) z roku 1990.

  \item \emph{Clinical Dementia Rating} (w skrócie \emph{CDR}) -- skala numeryczna używana do ilościowego określenia nasilenia się objawów demencyjnych i skategoryzowania choroby w określonym "stadium".
        Wynik w tej skali otrzymuje się jako rezultat przeprowadzenia ustrukturyzowanego protokołem wywiadu z pacjentem opracowanego w 1993 roku przez Morrisa i jego współpracowników, lekarzy z Washington University School of Medicine \cite{morris1993clinical}.

        CDR uważany jest za test zdolny do rozpoznania bardzo łagodnych upośledzeń, co przeważnie nie było możliwe w poprzednich testach klinicznych dotyczących choroby Alzheimera.
        Ma on jednak słabe strony, w tym najważniejszy -- długi czas potrzebny na przeprowadzenie całościowego wywiadu, i co za tym idzie jest względnie niezdolny do uchwycenia zmian w czasie.
        Dodatkowo polega on na subiektywnej ocenie lekarza, co może prowadzić do błędów.

  \item \emph{Alzheimer's Disease Assessment Scale -- Cognitive Subscale} (w skrócie \emph{ADAS--Cog}) -- test kliniczny, który mierzy zmiany w funkcjonowaniu poznawczym pacjenta i nasilenie objawów demencji.
        Jest połową większego testu, \emph{Alzheimer's Disease Assessment Scale} (w skrócie \emph{ADAS}), który składa się z dwóch części: \emph{ADAS--Cog} oraz \emph{ADAS--Noncog}.
        ADAS--Cog zawiera 11 pytań, które oceniają pamięć, orientację, rozumienie, umiejętności wykonawcze i język.
        Do dnia dzisiejszego jest on na szeroką skalę używany w badaniach klinicznych nad lekami na demencję czy chorobę Alzheimera i uważany za najlepszy standard  ocenie leczenia przeciw otępieniu \cite{connor2008administration}.

\end{itemize}

\section{Narzędzia i techniki uczenia maszynowego}

Narzędzia i techniki uczenia maszynowego

\section{Istniejące podejścia do detekcji choroby Alzheimera z użyciem uczenia maszynowego}

Istniejące podejścia do detekcji choroby Alzheimera z użyciem uczenia maszynowego
