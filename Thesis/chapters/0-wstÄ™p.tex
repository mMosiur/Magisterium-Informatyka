\chapter*{Wstęp}
\addcontentsline{toc}{chapter}{Wstęp}

Wśród wyzwań współczesnej medycyny i~informatyki, niezmiernie istotnym tematem stała się detekcja chorób neurodegeneracyjnych, zwłaszcza takiej jak choroba Alzheimera czy szerzej pojęta demencja.
Rozwój technologii i~zdolność do analizy ogromnych ilości danych stworzyły nowe możliwości w~dziedzinie diagnostyki i~przewidywania tych złożonych schorzeń.
Przy jednoczesnym użyciu potencjału narzędzi uczenia maszynowego oraz środowiska \emph{.NET}, otwierają się przed nami perspektywy badawcze, które mogą doprowadzić do istotnych postępów w~zrozumieniu oraz wczesnym wykrywaniu choroby Alzheimera.

Choroba Alzheimera, stanowiąca formę demencji, jest poważnym wyzwaniem dla opieki zdrowotnej na całym świecie.
Jej postępujący charakter i~wpływ na jakość życia pacjentów oraz ich bliskich stanowią bodziec do poszukiwania nowych narzędzi diagnostycznych.
Jest ona też najczęstszą przyczyną demencji, stanowiąc około $60$-$80$\% wszystkich stwierdzonych przypadków \cite{what-is-alzheimers:2023}.
Przeszłość medycyny skupiła się na metodach klinicznych i~obrazowaniu mózgu, jednak ostatnie lata przyniosły rewolucję w~zastosowaniu uczenia maszynowego do analizy medycznych danych.
W szczególności, techniki głębokiego uczenia maszynowego, które wykorzystują sieci neuronowe, stały się obiecującym narzędziem do diagnozowania chorób neurodegeneracyjnych.
W ostatnich latach, wiele badań skupiło się na wykorzystaniu obrazowania mózgu do detekcji choroby Alzheimera.

Z oczywistych względów najczęściej wykorzystywane środowiska i~narzędzia programistyczne używane w~celu uczenia modeli przeznaczonych do diagnostyki to te same, które cieszą się największą popularnością uniwersalnie w~dziedzinie inżynierii danych i~uczenia maszynowego -- w~głównej mierze języki Python i~R.

Jednakże w~samych zastosowaniach i~oprogramowaniach w~medycynie najczęściej spotyka się inne języki inne platformy, w~dominującej części jest to Java oraz na drugiej pozycji \emph{.NET} i~język C\#.
Java jest popularna zarówno w środowiskach akademickich jak i~medycznych, a~co za tym idzie podejmowanych jest wiele prób prowadzenia rozwojowych badań i~testów wykorzystania uczenia maszynowego w~celach predykcyjnych i~diagnostycznych \cite{soman2005classification, nithya2017predictive, gobbel2014development, godara2016evaluation}.
.NET natomiast jest raczej negatywnie odbierany w~środowisku akademickim a~popularność natomiast osiągnął głównie w~sferze rozwiązań komercyjnych -- w~tym również w~wielu systemat medycznych i~szpitalnych.
Więc mimo potencjału integracji rozwiązań uczenia maszynowego z~systemami medycznymi, wciąż uczenie maszynowe w~\emph{.NET} nie jest tematem często spotykanym.
Dlatego też postanowiłem sprawdzić, jakie możliwości daje nam platforma .NET w~kontekście uczenia maszynowego w~zastosowaniu do diagnostyki choroby Alzheimera.

Przyjrzenie się technologii głębokiego uczenia maszynowego oraz ich implementacji w~środowisku .NET jawi się jako kluczowe.
Platforma \emph{.NET}, z~jej wszechstronnymi językami programowania oraz narzędziami, stanowi wyjątkowe miejsce gdzie, w~różnorodnych rozwiązaniach wdrożyć można dodatkowe atuty płynące z~uczenia maszynowego.
Skoncentrowanie się na narzędziu \emph{ML.NET} oraz bibliotece \emph{TensorFlow.NET} umożliwia nam zbadanie najpopularniejszych i~najbardziej rozwiniętych podejść do uczenia modeli diagnostycznych w~kontekście choroby Alzheimera.

Celem niniejszej pracy jest nie tylko zgłębienie wiedzy na temat choroby Alzheimera oraz jej objawów, ale przede wszystkim wykorzystanie osiągnięć dziedziny uczenia maszynowego w~celu opracowania skutecznych modeli diagnostycznych.
Skonfrontowanie różnych podejść i~narzędzi pozwoli na lepsze zrozumienie procesu detekcji tej choroby oraz wydobycie potencjalnych strategii poprawy skuteczności i~efektywności diagnozowania.

W dalszych sekcjach pracy, zagłębimy się w~przeszłość i~teraźniejszość choroby Alzheimera, omówimy kluczowe metody diagnostyki oraz zaprezentujemy narzędzia i~technologie, które wykorzystamy do naszych badań porównawczych a~także ich wyniki i~wnioski.
