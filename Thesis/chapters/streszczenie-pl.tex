\begin{abstract}

  Choroba Alzheimera jest najbardziej powszechną chorobą neurodegeneracyjną dotykającą współczesne społeczeństwo.
  Jej skutki a~także skutki szerszej pojęciowo demencji są również bardzo dotkliwe i~szczególne w~sposobie, w~jaki dotykają zarówno chorych jak i~ich bliskich.
  W~celu rozwoju perspektyw wczesnego wykrywania choroby w~niniejszej pracy przedstawiam możliwości wykorzystania metod uczenia maszynowego do detekcji choroby Alzheimera oraz stopnia demencji jej towarzyszącej.
  Wykorzystuję w~tym celu narzędzia uczenia maszynowego dostępne w~środowisku \emph{.NET}, które zostało całkowicie pominięte przez środowisko akademickie, natomiast jest bardzo popularne w~systemach medycznych czy szpitalnych, gdzie potencjalnie tego typu systemy diagnostyczne mogą być wykorzystywane.
  Porównuję dwa popularne narzędzia uczenia maszynowego -- bibliotekę \emph{ML.NET} oraz \emph{TensorFlow.NET}, które wykorzystuję w~celu przeprowadzenia uczenia transferowego oraz uczenia głębokich konwolucyjnych sieci neuronowych, które uczę i~testuję na zbiorze obrazów rezonansu magnetycznego mózgu osób zdrowych oraz z~chorobą Alzheimera i~o różnym stopniu demencji.
  Najlepszy wytrenowany model osiąga dokładność na poziomie $85\%$ dla klasyfikacji stopnia demencji, które nie jest wynikiem złym, choć alternatywne rozwiązania powstające w~ostatnich latach osiągają już znacznie lepsze wyniki.
  Przedstawiam także wady i~zalety wykorzystanych narzędzi oraz możliwości ich rozwoju w~przyszłości.

\end{abstract}
